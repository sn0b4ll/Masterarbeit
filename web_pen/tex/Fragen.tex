\documentclass[11pt,DIV=11]{scrartcl} % 
\usepackage{amssymb}
\usepackage{array}
\usepackage{tabularx}
\usepackage[backend=biber,]{biblatex}
\usepackage[utf8]{luainputenc}

\renewcommand{\familydefault}{\sfdefault}
\usepackage{graphicx}

\usepackage{scrpage2} 	% Kopf & Fußzeile im KOMA Stil
\pagestyle{scrheadings}	% Aktiviert Verwendung vordefinierter Kolumnentitel
\clearscrheadfoot 		% alle Standard-Werte und Formatierungen löschen
\setkomafont{pagehead}{}	% Schriftart in Kopfzeile, \scshape = Kapitelchen
\automark[chapter]{section} % [linke Seite]{rechte Seite}
%\ohead{\def\pagestyle{PDTS}{\hrulefill\includegraphics[width = 6cm]{bilder/thi_logo_quer_cropped}}}
\ohead{\includegraphics[width = 3cm]{allianz_logo.jpg}}
\ihead{Fragebogen Penetrations-Test}

\setlength{\headsep}{7mm}		% Textabstand zur Kopfzeile
\setlength{\footskip}{7mm}		% Abstand zur Fußzeile

\ofoot{\vspace{-0.3cm} \pagemark} 
						
\setheadsepline{.2pt}
\setfootsepline{.4pt}	% Trennlinie Fußzeile und Textkörper

\newcolumntype{S}{>{\centering\arraybackslash}m{2em}}

\ifoot{\vspace{-1.7cm}
\begin{tabular}{l || l}
% Change me!
\textit{Unternehmen} 			& \textit{Pentester}\\ 
\textit{Straße Nr}			 	& \textit{Tel} \\
\textit{PLZ Ort} 					& \textit{E-Mail}\\
\textit{USTID}						& \textit{Webseite}
\end{tabular}
}
\newcommand{\frage}[1]{\makebox[\textwidth]{%
\renewcommand{\arraystretch}{2.0}
\begin{tabularx}{\textwidth}{|X|}
  \hline
  #1 \\
  \hline
  \line(1,0){420}\\
  \hline \hline
\end{tabularx}%
\renewcommand{\arraystretch}{1.0}
}}

\newcommand{\frageJaNein}[1]{\makebox[\textwidth]{%
\renewcommand{\arraystretch}{2.0}
\begin{tabularx}{\textwidth}{|X|S|S|}
  \hline
  #1 & Ja & Nein\\
  \hline
  \line(1,0){350} & $\square$ & $\square$	\\
  \hline \hline
\end{tabularx}%
\renewcommand{\arraystretch}{1.0}
}}

\newcommand{\frageOpt}[4]{\makebox[\textwidth]{%
\renewcommand{\arraystretch}{2.0}
\begin{tabularx}{\textwidth}{|X|X|X|}
  \hline
  \multicolumn{3}{|c|}{#1}\\
  \hline
  #2 $\square$ & #3 $\square$ & #4 $\square$\\
  \hline \hline
\end{tabularx}%
\renewcommand{\arraystretch}{1.0}
}}

\newcommand{\frageJaNeinKurz}[1]{\makebox[\textwidth]{%
\renewcommand{\arraystretch}{1.5}
\begin{tabularx}{\textwidth}{|X|S|S|}
  \hline
  #1 & Ja $\square$ & Nein $\square$\\
  \hline \hline
\end{tabularx}%
\renewcommand{\arraystretch}{1.0}
}}
\usepackage[parfill]{parskip}
\begin{document}
\section{Allgemeines}
\subsection{Ansprechpartner}
%%allg_anspr_web_app

\subsection{Eingrenzung}
Welche Art/en von Pentest/s sollen durchgeführt werden?
\begin{itemize}
	\item[%%allg_pen_art_wapt] Web-Application
	\item[%%allg_pen_art_npt] Network
	\item[%%allg_pen_art_se] Social Engineering
	\item[%%allg_pen_art_wl] Wireless
	\item[%%allg_pen_art_phys] Physical
\end{itemize}

\subsection{Allgemeine Fragen}\label{ref:allg_frag}
\makebox[\textwidth]{%
\renewcommand{\arraystretch}{2.0}
\begin{tabularx}{\textwidth}{|X|S|S|}
  \hline
  Ist der Test für eine spezielle Compliance-Anforderung notwendig? & Ja & Nein\\
  \hline
  %%alg_compliance	\\
  \hline \hline
\end{tabularx}%
\renewcommand{\arraystretch}{1.0}
}\\

\makebox[\textwidth]{%
\renewcommand{\arraystretch}{2.0}
\begin{tabularx}{\textwidth}{|X|}
  \hline
  Wann soll der Test statt finden? \\
  \hline
  %%allg_termin \\
  \hline \hline
\end{tabularx}%
\renewcommand{\arraystretch}{1.0}
}

\makebox[\textwidth]{%
\renewcommand{\arraystretch}{2.0}
\begin{tabularx}{\textwidth}{|X|X|X|}
  \hline
  \multicolumn{3}{|c|}{In welchen Zeiträumen soll der Test durchgeführt werden?}\\
  \hline
  %%alg_zeitraum \\
  \hline \hline
\end{tabularx}%
\renewcommand{\arraystretch}{1.0}
}
\newpage

\section{Web Application Penetration Test}\label{ref:WebAppPenTest}

\makebox[\textwidth]{%
\renewcommand{\arraystretch}{2.0}
\begin{tabularx}{\textwidth}{|X|S|S|}
  \hline
  Wird der Quellcode der Applikation/Webseite zugänglich gemacht? & Ja & Nein\\
  \hline
  %%wapt_quell_zug	\\
  \hline \hline
\end{tabularx}%
\renewcommand{\arraystretch}{1.0}
}

\makebox[\textwidth]{%
\renewcommand{\arraystretch}{2.0}
\begin{tabularx}{\textwidth}{|X|}
  \hline
  Wie viele Web-Applikationen sind In-Scope? \\
  \hline
  %%wapt_anz_web_app \\
  \hline \hline
\end{tabularx}%
\renewcommand{\arraystretch}{1.0}
}

\makebox[\textwidth]{%
\renewcommand{\arraystretch}{2.0}
\begin{tabularx}{\textwidth}{|X|}
  \hline
  Wie viele Login-Systeme sind In-Scope? \\
  \hline
  %%wapt_anz_login_sys \\
  \hline \hline
\end{tabularx}%
\renewcommand{\arraystretch}{1.0}
}

\makebox[\textwidth]{%
\renewcommand{\arraystretch}{2.0}
\begin{tabularx}{\textwidth}{|X|}
  \hline
  Wie viele statische Seiten sind ca. In-Scope? \\
  \hline
  %%wapt_anz_stat_seiten \\
  \hline \hline
\end{tabularx}%
\renewcommand{\arraystretch}{1.0}
}

\makebox[\textwidth]{%
\renewcommand{\arraystretch}{2.0}
\begin{tabularx}{\textwidth}{|X|}
  \hline
  Wie viele dynamische Seiten sind ca. In-Scope? \\
  \hline
  %%wapt_anz_dyn_seiten \\
  \hline \hline
\end{tabularx}%
\renewcommand{\arraystretch}{1.0}
}

\makebox[\textwidth]{%
\renewcommand{\arraystretch}{2.0}
\begin{tabularx}{\textwidth}{|X|S|S|}
  \hline
  Soll Fuzzing gegen die Applikation/en eingesetzt werden? & Ja & Nein\\
  \hline
  %%wapt_fuzzing	\\
  \hline \hline
\end{tabularx}%
\renewcommand{\arraystretch}{1.0}
}

\makebox[\textwidth]{%
\renewcommand{\arraystretch}{2.0}
\begin{tabularx}{\textwidth}{|X|S|S|}
  \hline
  Soll der Penetrations-Test aus verschiedenen Rollen durchgeführt werden? & Ja & Nein\\
  \hline
  %%wapt_vers_rollen	\\
  \hline \hline
\end{tabularx}%
\renewcommand{\arraystretch}{1.0}
}

\makebox[\textwidth]{%
\renewcommand{\arraystretch}{2.0}
\begin{tabularx}{\textwidth}{|X|S|S|}
  \hline
  Sollen Password-Scans auf die Webseite durchgeführt werden? & Ja & Nein\\
  \hline
  %%wapt_pwd_scan	\\
  \hline \hline
\end{tabularx}%
\renewcommand{\arraystretch}{1.0}
}

\section{Network Penetration Test}\label{ref:NetPenTest}

\makebox[\textwidth]{%
\renewcommand{\arraystretch}{2.0}
\begin{tabularx}{\textwidth}{|X|}
  \hline
  Was ist das Ziel des Penetrations-Test? \\
  \hline
  %%npt_ziel \\
  \hline \hline
\end{tabularx}%
\renewcommand{\arraystretch}{1.0}
}

\makebox[\textwidth]{%
\renewcommand{\arraystretch}{2.0}
\begin{tabularx}{\textwidth}{|X|}
  \hline
  Wie viele IP-Adressen sollen getestet werden? \\
  \hline
  %%npt_anz_ips \\
  \hline \hline
\end{tabularx}%
\renewcommand{\arraystretch}{1.0}
}

\makebox[\textwidth]{%
\renewcommand{\arraystretch}{2.0}
\begin{tabularx}{\textwidth}{|X|S|S|}
  \hline
  Sind Techniken im Einsatz, die die Resultate verfälschen könnten? (WAF, IPS etc.?) & Ja & Nein\\
  \hline
  %%npt_verteidigung	\\
  \hline \hline
\end{tabularx}%
\renewcommand{\arraystretch}{1.0}
}

\makebox[\textwidth]{%
\renewcommand{\arraystretch}{2.0}
\begin{tabularx}{\textwidth}{|X|}
  \hline
  Wie ist das Vorgehen bei einem gelungenen Angriff? \\
  \hline
  %%npt_gel_ang \\
  \hline \hline
\end{tabularx}%
\renewcommand{\arraystretch}{1.0}
}

\makebox[\textwidth]{%
\renewcommand{\arraystretch}{2.0}
\begin{tabularx}{\textwidth}{|X|S|S|}
  \hline
  Soll versucht werden lokale Admin-Rechte zu erlangen und tiefer in das Netz vorzudringen? & Ja & Nein\\
  \hline
  %%npt_nutz_lok_admin	\\
  \hline \hline
\end{tabularx}%
\renewcommand{\arraystretch}{1.0}
}

\makebox[\textwidth]{%
\renewcommand{\arraystretch}{2.0}
\begin{tabularx}{\textwidth}{|X|S|S|}
  \hline
  Sollen Angriffe auf gefundene Passwort-Hashes durchgeführt werden? & Ja & Nein\\
  \hline
  %%npt_pwd_hashes	\\
  \hline \hline
\end{tabularx}%
\renewcommand{\arraystretch}{1.0}
}

\section{Social Engineering}\label{ref:SocEngi}

\makebox[\textwidth]{%
\renewcommand{\arraystretch}{2.0}
\begin{tabularx}{\textwidth}{|X|S|S|}
  \hline
  Gibt es eine vollständige Liste von E-Mail-Adressen, die für den Test verwendet werden können? & Ja & Nein\\
  \hline
  %%se_emails	\\
  \hline \hline
\end{tabularx}%
\renewcommand{\arraystretch}{1.0}
}

\makebox[\textwidth]{%
\renewcommand{\arraystretch}{2.0}
\begin{tabularx}{\textwidth}{|X|S|S|}
  \hline
  Gibt es eine vollständige Liste von Telefon-Nummern, die für den Test verwendet werden können? & Ja & Nein\\
  \hline
  %%se_tels	\\
  \hline \hline
\end{tabularx}%
\renewcommand{\arraystretch}{1.0}
}

\makebox[\textwidth]{%
\renewcommand{\arraystretch}{2.0}
\begin{tabularx}{\textwidth}{|X|S|S|}
  \hline
  Ist das Einsetzen von Social Engineering zum Überwinden physikalischer Sicherheitseinrichtungen erlaubt? & Ja & Nein\\
  \hline
  %%se_phys	\\
  \hline \hline
\end{tabularx}%
\renewcommand{\arraystretch}{1.0}
}

\makebox[\textwidth]{%
\renewcommand{\arraystretch}{2.0}
\begin{tabularx}{\textwidth}{|X|}
  \hline
  Wie viele Personen sollen ca. getestet werden? \\
  \hline
  %%se_anz_pers \\
  \hline \hline
\end{tabularx}%
\renewcommand{\arraystretch}{1.0}
}

\section{Wireless Network Penetration Test}\label{ref:WirNetPen}

\makebox[\textwidth]{%
\renewcommand{\arraystretch}{2.0}
\begin{tabularx}{\textwidth}{|X|}
  \hline
  Wieviele Funk-Netzwerke sind im Einsatz? \\
  \hline
  %%wl_anz_fnw \\
  \hline \hline
\end{tabularx}%
\renewcommand{\arraystretch}{1.0}
}

\makebox[\textwidth]{%
\renewcommand{\arraystretch}{2.0}
\begin{tabularx}{\textwidth}{|X|S|S|}
  \hline
  Gibt es eine Gäste WLAN? Wenn ja, wie ist dieses Umgesetzt? & Ja & Nein\\
  \hline
  %%wl_gastwlan	\\
  \hline \hline
\end{tabularx}%
\renewcommand{\arraystretch}{1.0}
}

\makebox[\textwidth]{%
\renewcommand{\arraystretch}{2.0}
\begin{tabularx}{\textwidth}{|X|}
  \hline
  Welche Verschlüsselung wird für die Netzwerke genutzt? \\
  \hline
  %%wl_encr \\
  \hline \hline
\end{tabularx}%
\renewcommand{\arraystretch}{1.0}
}

\makebox[\textwidth]{%
\renewcommand{\arraystretch}{2.0}
\begin{tabularx}{\textwidth}{|X|S|S|}
  \hline
  Sollen nicht-firmen-Geräte im WLAN aufgespürt werden? & Ja & Nein\\
  \hline
  %%wl_fremgeraet	\\
  \hline \hline
\end{tabularx}%
\renewcommand{\arraystretch}{1.0}
}

\makebox[\textwidth]{%
\renewcommand{\arraystretch}{2.0}
\begin{tabularx}{\textwidth}{|X|S|S|}
  \hline
  Soll Netz-Attacken gegen Clients durchgeführt werden? & Ja & Nein\\
  \hline
  %%wl_att_clients	\\
  \hline \hline
\end{tabularx}%
\renewcommand{\arraystretch}{1.0}
}

\makebox[\textwidth]{%
\renewcommand{\arraystretch}{2.0}
\begin{tabularx}{\textwidth}{|X|}
  \hline
  Wie viele Clients nutzen das WLAN ca.? \\
  \hline
  %%wl_clients \\
  \hline \hline
\end{tabularx}%
\renewcommand{\arraystretch}{1.0}
}

\section{Physical Penetration Test}\label{ref:PhysPen}

\makebox[\textwidth]{%
\renewcommand{\arraystretch}{2.0}
\begin{tabularx}{\textwidth}{|X|}
  \hline
  Wie viele Einrichtungen sollen getestet werden? \\
  \hline
  %%phys_anz_einr \\
  \hline \hline
\end{tabularx}%
\renewcommand{\arraystretch}{1.0}
}

\makebox[\textwidth]{%
\renewcommand{\arraystretch}{2.0}
\begin{tabularx}{\textwidth}{|X|S|S|}
  \hline
  Wird die Einrichtung mit anderen Parteien geteilt? & Ja & Nein\\
  \hline
  %%phys_anz_part	\\
  \hline \hline
\end{tabularx}%
\renewcommand{\arraystretch}{1.0}
}

\makebox[\textwidth]{%
\renewcommand{\arraystretch}{2.0}
\begin{tabularx}{\textwidth}{|X|S|S|}
  \hline
  Muss Sicherheitspersonal umgangen werden? & Ja & Nein\\
  \hline
  %%phys_sich_pers \\
  \hline \hline
\end{tabularx}%
\renewcommand{\arraystretch}{1.0}
}

\makebox[\textwidth]{%
\renewcommand{\arraystretch}{2.0}
\begin{tabularx}{\textwidth}{|X|S|S|}
  \hline
  Wird das Sicherheitspersonal durch einen Dritten gestellt? & Ja & Nein\\
  \hline
  %%phys_sich_pers_dritt	\\
  \hline \hline
\end{tabularx}%
\renewcommand{\arraystretch}{1.0}
}

\makebox[\textwidth]{%
\renewcommand{\arraystretch}{2.0}
\begin{tabularx}{\textwidth}{|X|S|S|}
  \hline
  Ist das Sicherheitspersonal bewaffnet? & Ja & Nein\\
  \hline
  %%phys_sich_pers_waffen	\\
  \hline \hline
\end{tabularx}%
\renewcommand{\arraystretch}{1.0}
}

\makebox[\textwidth]{%
\renewcommand{\arraystretch}{2.0}
\begin{tabularx}{\textwidth}{|X|S|S|}
  \hline
  Ist der Einsatz von körperlicher Gewalt durch das Sicherheitspersonal gestattet? & Ja & Nein\\
  \hline
  %%phys_sich_pers_gewalt	\\
  \hline \hline
\end{tabularx}%
\renewcommand{\arraystretch}{1.0}
}

\makebox[\textwidth]{%
\renewcommand{\arraystretch}{2.0}
\begin{tabularx}{\textwidth}{|X|}
  \hline
  Wie viele Eingänge gibt es zu der/den Einrichtung/en? \\
  \hline
  %%phys_anz_eing \\
  \hline \hline
\end{tabularx}%
\renewcommand{\arraystretch}{1.0}
}

\makebox[\textwidth]{%
\renewcommand{\arraystretch}{2.0}
\begin{tabularx}{\textwidth}{|X|S|S|}
  \hline
  Ist das knacken von Schlössern oder fälschen von Schlüsseln erlaubt? & Ja & Nein\\
  \hline
  %%phys_knacken	\\
  \hline \hline
\end{tabularx}%
\renewcommand{\arraystretch}{1.0}
}

\makebox[\textwidth]{%
\renewcommand{\arraystretch}{2.0}
\begin{tabularx}{\textwidth}{|X|}
  \hline
  Wie groß ist die Fläche ungefähr? \\
  \hline
  %%phys_flaeche \\
  \hline \hline
\end{tabularx}%
\renewcommand{\arraystretch}{1.0}
}

\makebox[\textwidth]{%
\renewcommand{\arraystretch}{2.0}
\begin{tabularx}{\textwidth}{|X|S|S|}
  \hline
  Sind alle physikalischen Sicherheitsmaßnahmen dokumentiert und werden zur Verfügung gestellt? & Ja & Nein\\
  \hline
  %%phys_whitebox	\\
  \hline \hline
\end{tabularx}%
\renewcommand{\arraystretch}{1.0}
}

\makebox[\textwidth]{%
\renewcommand{\arraystretch}{2.0}
\begin{tabularx}{\textwidth}{|X|S|S|}
  \hline
  Werden Video-Kameras verwendet? & Ja & Nein\\
  \hline
  %%phys_cam	\\
  \hline \hline
\end{tabularx}%
\renewcommand{\arraystretch}{1.0}
}

\makebox[\textwidth]{%
\renewcommand{\arraystretch}{2.0}
\begin{tabularx}{\textwidth}{|X|S|S|}
  \hline
  Werden diese Kameras durch Dritte verwaltet? & Ja & Nein\\
  \hline
  %%phys_cam_dritte	\\
  \hline \hline
\end{tabularx}%
\renewcommand{\arraystretch}{1.0}
}

\makebox[\textwidth]{%
\renewcommand{\arraystretch}{2.0}
\begin{tabularx}{\textwidth}{|X|S|S|}
  \hline
  Soll versucht werden, die aufgezeichneten Daten zu löschen? & Ja & Nein\\
  \hline
  %%phys_loeschen	\\
  \hline \hline
\end{tabularx}%
\renewcommand{\arraystretch}{1.0}
}

\makebox[\textwidth]{%
\renewcommand{\arraystretch}{2.0}
\begin{tabularx}{\textwidth}{|X|S|S|}
  \hline
  Gibt es ein Alarm-System? & Ja & Nein\\
  \hline
  %%phys_alarm	\\
  \hline \hline
\end{tabularx}%
\renewcommand{\arraystretch}{1.0}
}

\makebox[\textwidth]{%
\renewcommand{\arraystretch}{2.0}
\begin{tabularx}{\textwidth}{|X|S|S|}
  \hline
  Gibt es einen Stillen Alarm? & Ja & Nein\\
  \hline
  %%phys_alarm_still	\\
  \hline \hline
\end{tabularx}%
\renewcommand{\arraystretch}{1.0}
}

\makebox[\textwidth]{%
\renewcommand{\arraystretch}{2.0}
\begin{tabularx}{\textwidth}{|X|}
  \hline
  Welche Ereignisse lösen den Alarm aus? \\
  \hline
  %%phys_alarm_ausloeser \\
  \hline \hline
\end{tabularx}%
\renewcommand{\arraystretch}{1.0}
}

\section{Questions for Systems Administrators}\label{ref:QuesSysAdmin}

\makebox[\textwidth]{%
\renewcommand{\arraystretch}{2.0}
\begin{tabularx}{\textwidth}{|X|S|S|}
  \hline
  Gibt es Systeme, die als instabil angesehen werden (alte Patch-Stände, Legacy Systeme etc.)? & Ja & Nein\\
  \hline
  %%sysadm_altsysteme	\\
  \hline \hline
\end{tabularx}%
\renewcommand{\arraystretch}{1.0}
}

\makebox[\textwidth]{%
\renewcommand{\arraystretch}{2.0}
\begin{tabularx}{\textwidth}{|X|S|S|}
  \hline
  Gibt es Systeme von Dritten, die ausgeschlossen werden müssen oder für die weitere Genehmigungen notwendig sind? & Ja & Nein\\
  \hline
  %%sysadm_drittsysteme	\\
  \hline \hline
\end{tabularx}%
\renewcommand{\arraystretch}{1.0}
}

\makebox[\textwidth]{%
\renewcommand{\arraystretch}{2.0}
\begin{tabularx}{\textwidth}{|X|}
  \hline
  Was ist die Durchschnittszeit zur Wiederherstellung der Funktionalität eines Services? \\
  \hline
  %%sysadm_wiederherstellungszeit \\
  \hline \hline
\end{tabularx}%
\renewcommand{\arraystretch}{1.0}
}

\makebox[\textwidth]{%
\renewcommand{\arraystretch}{2.0}
\begin{tabularx}{\textwidth}{|X|S|S|}
  \hline
  Ist eine Monitoring-Software im Einsatz? & Ja & Nein\\
  \hline
  %%sysadm_monitoring	\\
  \hline \hline
\end{tabularx}%
\renewcommand{\arraystretch}{1.0}
}

\makebox[\textwidth]{%
\renewcommand{\arraystretch}{2.0}
\begin{tabularx}{\textwidth}{|X|}
  \hline
  Welche sind die kritischsten Applikationen? \\
  \hline
  %%sysadm_krit_appl \\
  \hline \hline
\end{tabularx}%
\renewcommand{\arraystretch}{1.0}
}

\makebox[\textwidth]{%
\renewcommand{\arraystretch}{2.0}
\begin{tabularx}{\textwidth}{|X|S|S|}
  \hline
  Werden in einem regelmäßigen Turnus Backups erstellt und getestet? & Ja & Nein\\
  \hline
  %%sysadm_backup	\\
  \hline \hline
\end{tabularx}%
\renewcommand{\arraystretch}{1.0}
}

\section{Questions for Business Unit Managers}\label{ref:QuesBUM}

\makebox[\textwidth]{%
\renewcommand{\arraystretch}{2.0}
\begin{tabularx}{\textwidth}{|X|S|S|}
  \hline
  Ist die Führungsebene über den Test informiert? & Ja & Nein\\
  \hline
  %%bum_mgmt_inform	\\
  \hline \hline
\end{tabularx}%
\renewcommand{\arraystretch}{1.0}
}

\makebox[\textwidth]{%
\renewcommand{\arraystretch}{2.0}
\begin{tabularx}{\textwidth}{|X|}
  \hline
  Welche Daten stellen das größte Risiko dar, falls diese manipuliert werden? \\
  \hline
  %%bum_krit_daten \\
  \hline \hline
\end{tabularx}%
\renewcommand{\arraystretch}{1.0}
}

\makebox[\textwidth]{%
\renewcommand{\arraystretch}{2.0}
\begin{tabularx}{\textwidth}{|X|S|S|}
  \hline
  Gibt es Testfälle, die die Funktionalität der Services prüfen und belegen können? & Ja & Nein\\
  \hline
  %%bum_monitoring	\\
  \hline \hline
\end{tabularx}%
\renewcommand{\arraystretch}{1.0}
}

\makebox[\textwidth]{%
\renewcommand{\arraystretch}{2.0}
\begin{tabularx}{\textwidth}{|X|S|S|}
  \hline
  Sind "Disaster Recovery Procedures" vorhanden? & Ja & Nein\\
  \hline
  %%bum_disaster_recovery	\\
  \hline \hline
\end{tabularx}%
\renewcommand{\arraystretch}{1.0}
}

\end{document}