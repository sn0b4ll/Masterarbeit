
%------------------------------------------------------------------------
% Anhänge
\appendix
\chapter{Quellcode}

\section{Fragebogen}\label{ap:Fragebogen}
\paragraph{Allgemein}
\begin{itemize}
	\item Wie ist der Anwendungs/Projekt-Name?
	\item Wer sind die Ansprechpartner?
	\item Welche Art/en von Pen-Test/s sollen durchgeführt werden? \\
	(Web-Application/Web-Service/Mobile-Application/Network/Social Engineering/Wireless/Physical)
	\item Ist der Test für eine spezielle Compliance-Anforderung notwendig?\\
	(Ja/Nein)
	\item Wann soll der Test durchgeführt werden?
	\item In welchen Zeiträumen soll der Test durchgeführt werden?\\
	(Bürozeiten/Feierabend/Wochenende)
\end{itemize}

\paragraph{Web Application Penetration Test}
\begin{itemize}
	\item Wie geschieht der Zugriff auf die Anwendung?\\
	(Aus dem Internet erreichbar/IP-Einschränkung/VPN/Pen-Test muss intern durchgeführt werden)
	\item Welche Funktionalitäten hat die Anwendung?\\
	(CMS/Captcha/Upload/Download/Browser-Plugins/Workflows)
	\item In welcher Stage befindet sich die Anwendung?\\
	(Development oder Test/System Integration/Produktion)
	\item Wird der Quellcode der Applikation/Webseite zugänglich gemacht?\\
	(Ja/Nein)
	\item Wie viele Web-Applikationen sind In-Scope?
	\item Wie viele Login-Systeme sind In-Scope?
	\item Wie viele statische Seiten sind ca. In-Scope?
	\item Wie viele dynamische Seiten sind ca. In-Scope?
	\item Soll Fuzzing gegen die Applikation/en eingesetzt werden?\\
	(Ja/Nein)
	\item Soll der Penetration-Test aus verschiedenen Rollen durchgeführt werden?\\
	(Ja/Nein)
	\item  Wie ist die Anmeldung gestaltet?\\
	(Benutzername und Passwort/Zertifikat/Komplexeres System)
	\item Welche Technologien nutzt die Anwendung?
	\item Sollen Password-Scans auf die Webseite durchgeführt werden?\\
	(Ja/Nein)
\end{itemize}

\paragraph{Web Service Penetration Test}
\begin{itemize}
	\item Wie geschieht der Zugriff auf die Anwendung?\\
	(Aus dem Internet erreichbar/IP-Einschränkung/VPN/Pen-Test muss intern durchgeführt werden)
	\item In welcher Stage befindet sich die Anwendung?\\
	(Development oder Test/System Integration/Produktion)
	\item Wird der Quellcode des Services zugänglich gemacht? \\
	(Ja/Nein)
	\item Wie viele Web-Services sind In-Scope?
	\item Soll Fuzzing gegen die Applikation/en eingesetzt werden?\\
	(Ja/Nein)
	\item Soll der Penetration-Test aus verschiedenen Rollen durchgeführt werden?\\
	(Ja/Nein)
	\item  Wie ist die Anmeldung gestaltet?\\
	(Benutzername und Passwort/Zertifikat/Komplexeres System)
	\item Welche Technologien wurden für den Service genutzt??
	\item Sollen Password-Scans auf den Web-Service durchgeführt werden?\\
	(Ja/Nein)
\end{itemize}

\paragraph{Mobile Application Penetration Test}
\begin{itemize}
	\item Welche Technologien wurden für die App genutzt?
	\item Wird der Quellcode der App zugänglich gemacht?\\
	(Ja/Nein)
	\item Gibt es eine Root-Detection? Wenn ja, welche Konsequenzen hat eine positive Erkennung?\\
	(Ja/Nein)
	\item Hat die App einen Login?\\
	(Ja/Nein)
	\item Soll der Penetration-Test aus verschiedenen Rollen durchgeführt werden?\\
	(Ja/Nein)
	\item Soll Fuzzing gegen die App eingesetzt werden?\\
	(Ja/Nein)
\end{itemize}

\paragraph{Network Penetration Test}
\begin{itemize}
	\item Was ist das Ziel des Penetration-Test?
	\item Wie viele IP-Adressen sollen getestet werden?
	\item Sind Techniken im Einsatz, die die Resultate verfälschen könnten? (WAF, IPS etc.?)\\
	(Ja/Nein)
	\item Wie ist das Vorgehen bei einem gelungenen Angriff?
	\item Soll versucht werden lokale Admin-Rechte zu erlangen und tiefer in das Netz vorzudringen?\\
	(Ja/Nein)
	\item Sollen Angriffe auf gefundene Passwort-Hashes durchgeführt werden?\\
	(Ja/Nein)
\end{itemize}

\paragraph{Social Engineering}
\begin{itemize}
	\item Gibt es eine vollständige Liste von E-Mail-Adressen, die für den Test verwendet werden können?\\
	(Ja/Nein)
	\item Gibt es eine vollständige Liste von Telefon-Nummern, die für den Test verwendet werden können?\\
	(Ja/Nein)
	\item Ist das Einsetzen von Social Engineering zum Überwinden physikalischer Sicherheitseinrichtungen erlaubt?\\
	(Ja/Nein)
	\item Wie viele Personen sollen ca. getestet werden?
\end{itemize}

\paragraph{Wireless Network Penetration Test}
\begin{itemize}
	\item Wie viele Funk-Netzwerke sind im Einsatz?
	\item Gibt es ein Gäste WLAN? Wenn ja, wie ist dieses umgesetzt?\\
	(Ja/Nein)
	\item Welche Verschlüsselung wird für die Netzwerke genutzt?
	\item Sollen Nicht-Firmen-Geräte im WLAN aufgespürt werden?\\
	(Ja/Nein)
	\item Sollen Netz-Attacken gegen Clients durchgeführt werden?\\
	(Ja/Nein)
	\item Wie viele Clients nutzen das WLAN ca.?
\end{itemize}

\paragraph{Physical Penetration Test}
\begin{itemize}
	\item Wie viele Einrichtungen sollen getestet werden?
	\item Wird die Einrichtung mit anderen Parteien geteilt?\\
	(Ja/Nein)
	\item Muss Sicherheitspersonal umgangen werden?\\
	(Ja/Nein)
	\item Wird das Sicherheitspersonal durch einen Dritten gestellt?\\
	(Ja/Nein)
	\item Ist das Sicherheitspersonal bewaffnet?\\
	(Ja/Nein)
	\item Ist der Einsatz von körperlicher Gewalt durch das Sicherheitspersonal gestattet?\\
	(Ja/Nein)
	\item Wie viele Eingänge gibt es zu der/den Einrichtung/en?
	\item Ist das Knacken von Schlössern oder Fälschen von Schlüsseln erlaubt?\\
	(Ja/Nein)
	\item Wie groß ist die Fläche der Niederlassung ungefähr?
	\item Sind alle physikalischen Sicherheitsmaßnahmen dokumentiert und werden zur Verfügung gestellt?\\
	(Ja/Nein)
	\item Werden Video-Kameras verwendet?\\
	(Ja/Nein)
	\item Werden diese Kameras durch Dritte verwaltet?\\
	(Ja/Nein)
	\item Soll versucht werden, die aufgezeichneten Daten zu löschen?\\
	(Ja/Nein)
	\item Gibt es ein Alarm-System?\\
	(Ja/Nein)
	\item Gibt es einen Stillen Alarm?\\
	(Ja/Nein)
	\item Welche Ereignisse lösen den Alarm aus?
\end{itemize}

\section{Kickoff-Fragebogen}\label{ap:kickofffrag}
\paragraph{Allgemein}
\begin{itemize}
	\item Wie ist der Projekt-Name?
	\item Wer sind die Teilnehmer?
	\item Wann soll der Test durchgeführt werden?
	\item Was ist das Ziel des Tests?
	\item In welcher Stage befindet sich die Anwendung? Development Test System Integration Produktion
	\item Welche IP-Adressen sollen getestet werden?
	\item Welche URLs sollten getestet werden?
	\item Welche Zugangsdaten sollen genutzt werden?
	\item Sollen Denial-Of-Service-Angriffe durchgeführt werden?\\
	(Ja/Nein)
	\item Liegt ein Haftungsauschschluss vor?\\
	(Ja/Nein)
	\item Liegt eine Erlaubnis des Server-Betreibers vor?\\
	(Ja/Nein)
	\item Gibt es eine Deadline für den Bericht?\\
	(Ja/Nein)
	\item Erste Ergebnisse am Ende des Pen-Tests in einfacher Form zukommen lassen (z.B. Excel)?\\
	(Ja/Nein)
\end{itemize}


\paragraph{Fragen an den System-Administrator}
\begin{itemize}
\item Gibt es Systeme, die als instabil angesehen werden (alte Patch-Stände, Legacy Systeme etc.)?\\
(Ja/Nein)
\item Gibt es Systeme von Dritten, die ausgeschlossen werden müssen oder für die weitere Genehmigungen notwendig sind?\\
(Ja/Nein)
\item Was ist die Durchschnittszeit zur Wiederherstellung der Funktionalität eines Services?
\item Ist eine Monitoring-Software im Einsatz?\\
(Ja/Nein)
\item Welche sind die kritischsten Applikationen?
\item Werden in einem regelmäßigen Turnus Backups erstellt und getestet?\\
(Ja/Nein)
\end{itemize}

\paragraph{Fragen an den Business Unit Manager}
\begin{itemize}
\item Ist die Führungsebene über den Test informiert?\\
(Ja/Nein)
\item Welche Daten stellen das größte Risiko dar, falls diese manipuliert werden?
\item Gibt es Testfälle, die die Funktionalität der Services prüfen und belegen können?\\
(Ja/Nein)
\item Sind "`Disaster Recovery Procedures"' vorhanden?\\
(Ja/Nein)
\end{itemize}

\paragraph{Abschluss}
\begin{itemize}
\item Offene TODOs
\end{itemize}

\section{Pentest-Helper}
\subsection{Latex}\label{ap:FragTex}
\lstset{language=TeX}
\lstinputlisting[caption={packages.tex}]{./Fragenkatalog/packages.tex}
\lstinputlisting[caption={question\_templates.tex}]{./Fragenkatalog/question_templates.tex}
\lstinputlisting[caption={Frage.tex}]{./Fragenkatalog/Fragen.tex}

\subsection{Docker}\label{ap:Docker}
\lstinputlisting[caption={Dockerfile für Entwicklungsversion}]{./logs/Dockerfile}

\section{iOS-Simulator}
\subsection{Control-Script}\label{ap:simcontrol}
\lstset{language=Python}
\lstinputlisting[caption={control.py}]{./logs/control.py}
\subsection{MITM-Server}\label{ap:mitmserver}
\lstset{language=Python}
\lstinputlisting[caption={MITM-Server}]{./logs/mitm_server.py}

\section{MobSF}
%\subsection{BinSkim}label{ap:BinSkimRegeln}
%\lstinputlisting[caption={BinSkim-Regeln}]{./logs/20170213_binskim_rules.txt}

\subsection{BinScope-Installer}\label{ap:BinScopeInstaller}
\lstset{language=Python}
\lstinputlisting[caption={BinScope-Installer}]{./logs/binscope_install.py}

%\newpage
%\subsection{SAF.cgi}
%\lstinputlisting[caption={SAF.cgi}]{../../SAF/SAFConsole/web/cgi-bin/SAF.cgi}

%\subsection{framework}\label{code:framework}
%\lstinputlisting[caption={SAFSubRout.pm}]{../../SAF/SAFConsole/framework/SAFSubRout.pm}


%\newpage
%\subsection{HTML-Snippets}
%\lstset{language=HTML}
%\lstinputlisting[caption={htmlStart.snip}]{../../SAF/SAFConsole/web/cgi-bin/snippets/htmlStart.snip}
%\lstinputlisting[caption={htmlEnd.snip}]{../../SAF/SAFConsole/web/cgi-bin/snippets/htmlEnd.snip}

%\section{Module}
%\subsection{Cracking}
%\subsubsection{John the Ripper}
%\lstset{language=Perl}
%\lstinputlisting[caption={john.mod}]{../../SAF/SAFConsole/modules/cracking/john.mod}
%\lstinputlisting[caption={johnDemo.pl}]{../../SAF/SAFConsole/modules/cracking/john/johnDemo.pl}

%\subsection{Sniffing}
%\subsubsection{Ettercap-Bridged}
%\lstset{language=Perl}
%\lstinputlisting[caption={ettercap\_bridged.mod}]{../../SAF/SAFConsole/modules/sniffing/ettercap_bridged.mod}
%\lstinputlisting[caption={ettercap\_bridged\_demo.pl}]{../../SAF/SAFConsole/modules/sniffing/ettercap_bridged/ettercap_bridged_demo.pl}

%TODO Module, Config