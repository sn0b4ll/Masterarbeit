\chapter{Penetration-Tests mobiler Anwendungen}
Durch die zunehmende Nachfrage des Marktes nach mobilen Anwendungen entwickelt die Allianz Deutschland AG zunehmend mobile Applikationen. Dafür müssen nicht nur die bestehenden Prozesse angepasst, sondern auch neue Werkzeuge zur Unterstützung der Security-Prozesse geschaffen werden.\\

Im Folgenden sind die Anforderungen an ein solches Werkzeug festgehalten, gefolgt mit einer Evaluierung bestehender Programme. Daraufhin wurde ein passendes Werkzeug ausgewählt und weiterentwickelt. Diese Weiterentwicklung ist unter Abschnitt \ref{Weiterentwicklung MobSF} beschrieben. Abschließend werden die erreichten Anforderungen mit den ursprünglichen verglichen.

\section{Anforderungen}\label{ref:PenMobAnwdWeiterAnford}
Im Folgenden sind die Anforderungen festgehalten, welche für einen Echt-Einsatz in der Allianz Deutschland AG notwendig sind.

\begin{description}
	\item[Einfache Einrichtung] Das Tool sollte möglichst unkompliziert einem Benutzer zur Verfügung gestellt werden können. Dies könnte entweder über eine einfache Installation oder durch die Ausführung auf einer zentralen, erreichbaren Instanz realisiert werden.
	
	\item[Einfache Handhabung] Im Optimalfall soll ein Scan bereits nach einer kurzen Einführungszeit durch einen im Bereich Informationssicherheit zuvor ungeschulten oder nur rudimentär geschulten Mitarbeiter durchgeführt werden können. Dazu muss das Tool einfach zu bedienen sein.
	
	\item[Verständliche Ergebnisse] Es sollte eine Übersicht geben, die auch einen Mitarbeiter ohne Erfahrungen im Bereich Informationssicherheit eine erste Einschätzung ermöglicht. Die genauen Ergebnisse des Werkzeugs sollten für einen in der Informationssicherheit arbeitenden Angestellten verständlich sein. Hier kann eine gewisse Fachkenntnis vorausgesetzt werden. Grundlagen (wie zum Beispiel warum \textit{memcpy} auf eine mögliche Schwachstelle hinweist) müssen durch das Werkzeug nicht vermittelt werden.
	
	\item[Unterstützung für \textit{Android}/\textit{iOS}/\textit{Windows Phone}] Um zu verhindern, dass für jede der aktuell geläufigen Plattformen ein eigenes Werkzeug genutzt werden muss, sollte das Werkzeug die Analyse von Apps für \textit{Android}, \textit{iOS} und \textit{Windows Phone} unterstützen.
	
	\item[Niedrige False-Positive-Rate] Das Werkzeug sollte eine möglichst geringe Rate an \textit{False-Positives}, also falschen Meldungen, aufweisen. Jedoch sollten auch keine Hinweise auf Schwachstellen verworfen werden, sodass das Werkzeug eine Einstufung der Ergebnisse vornehmen sollte.
	
	\item[Reproduzierbarkeit] Das Werkzeug sollte reproduzierbare Ergebnisse liefern, also bei gleichem Eingangswert das gleiche Ergebnis generieren. Dies ist für die Nachverfolgung von Meldungen äußerst wichtig.
	
\end{description}

\section{Bestehende Anwendungen}
Trotz der relativ neuen Thematik der Mobilen Applikationen gibt es schon einige Programme und Applikationen, die bei der Identifizierung von Schwachstellen helfen können. Im Folgenden sind diese unterteilt in \textit{All-In-One-Framework} und Einzelanwendungen. Die Namen sind hierbei sprechend: Sogenannte \textit{All-In-One-Frameworks} bündeln mehrere kleine Anwendungen und automatisieren den Ablauf oder vereinfachen die Bedienung.

\subsection{All-In-One-Framework: MobSF}
\textit{MobSF} ist das einzige, derzeit öffentlich Verbreitete All-In-One-Framework zur Analyse von Mobilen Applikationen. Es ist eine Plattform zur statischen Analyse von Android und iOS-Apps sowie zur dynamischen Analyse von Android Apps. Es bündelt viele kleinere Anwendungen, welche unter \ref{Pen:Eingelanwendungen} aufgeführt sind, in einer einfachen Weboberfläche. Es ist Open-Source, in \textit{Python} geschrieben und steht in \textit{GIT} frei zur Verfügung.\footnote{\url{https://github.com/ajinabraham/Mobile-Security-Framework-MobSF}} Die aktuelle Version ist \textit{0.9.2 beta}.

Es unterstützt die statische Analyse von Apps in den Formaten \textit{APK} und \textit{IPA} sowie aus einfach komprimierten Archiven (\textit{zip}). Zusätzlich beinhaltet \textit{MobSF} einen eingebauten API Fuzzer und ist in der Lage, API-spezifische Schwachstellen wie XXE, SSRF oder Path Traversal zu erkennen (TODO Auflisten).

\subsection{Einzelanwendungen}\label{Pen:Eingelanwendungen}
Das All-In-One-Framework MobSF greift im Hintergrund oft auf eigenständige Tools zurück. Da es für Penetration-Test oft hilfreich ist, diese ohne ein umgebendes Framework nutzen zu können, sind im Folgenden die wichtigsten Tools kurz aufgeführt.

\subsubsection{•}

\newpage
\section{Vergleich der Emulationsumgebungen}\label{ref:VerglAktSitEmu}
Ein wichtiger Bestandteil in der dynamischen Analyse von Apps ist die Möglichkeit, Applikationen in einer emulierten Umgebung auszuführen. Im Folgenden werden diese Möglichkeiten für die \textit{iOS}, \textit{Windows-Phone} und \textit{Android} getestet.
 
	\subsection{iOS}
	Im Folgenden wurde speziell der in \textit{Xcode} enthaltene, offizielle \textit{iOS}-Simulator in seiner Funktionsweise untersucht.
	
			\subsubsection{Emulation}\label{ref:emulation}
			Die Emulation von \textit{iOS}-Geräten ist derzeit mit der Verwendung von \textit{Xcode} möglich. \textit{Xcode} wiederum ist nur unter \textit{Mac OS X} erhältlich. Da \textit{Max OS X} laut EULA nur auf "`Apple-branded computers"' verwendet werden darf \cite{AppleEULA}, ist die Simulation von \textit{iOS}-Geräten nur unter Apple-Hardware möglich. Nach der Installation über den in \textit{Mac OS X} enthaltenen App-Store kann ein virtualisiertes \textit{iPhone} über die Schritte \textit{XCode} $\rightarrow$ \textit{Open Developer Tools} $\rightarrow$ \textit{Simulator} oder über 
\begin{lstlisting}
/Applications/Xcode.app/Contents/Developer/Applications/Simulator.app
\end{lstlisting}			
gestartet werden.\\
			
		\subsubsection{Debugging}
Als Debugger unter \textit{Mac OS X} hat sich \textit{LLDB} etabliert und stellt das Pendant zu \textit{GDB} unter Linux dar. \textit{LLDB} ist kostenlos verfügbar, Open-Source und steht unter der University of Illinois/NCSA Open Source License\footnote{\url{https://opensource.org/licenses/UoI-NCSA.php}}, welche die Vervielfältigung und Veränderung des Quellcodes unter unter Hinweis auf \textit{LLVM} erlaubt.\\

\textit{LLDB} sollte auf jedem \textit{Mac OS X} System mit \textit{XCode} automatisch installiert sein und lässt sich im Terminal über das Kommando
\begin{lstlisting}
lldb
\end{lstlisting} aufrufen. Eine Gegenüberstellung von \textit{GDB}-Kommandos zu \textit{LLDB} steht auf der \textit{LLDB}-Webseite\footnote{\url{http://lldb.llvm.org/lldb-gdb.html}} zur Verfügung.\\

 Kompiliert man eine Applikation in \textit{xCode}, wird diese in einem emulierten \textit{iPhone} ge\-star\-tet und direkt ein Fenster \textit{LLDB} hergestellt. Die ausgeführte Applikation ist in \textit{LLDB} automatisch ausgewählt.\\

\begin{figure}[htbp]
	\centering
	\includegraphics[width=\textwidth]{bilder/pentest_mobile_anwendungen/vergleich_aktuelle_situation/20160627_XCode-LLDB.png}
	\caption{LLDB in XCode}
	\label{fig:LLDBinXCode}
\end{figure}
Ein Ziel dieser Arbeit ist jedoch das Automatisieren von Analysen, weshalb das Ausführen der grafischen Oberfläche nicht optimal ist.\\

Leider ist nicht erkennbar, wie \textit{LLDB} und das emulierte \textit{iPhone} eine Verbindung herstellen. Eine Auflistung der offenen Sockets auf dem System legt jedoch nahe, dass auf dem \textit{iPhone} das Programm \textit{debugserver} gestartet wird, welches Remote-Debugging mit \textit{LLDB} erlaubt. Es bleibt herauszufinden, wie die Debugging-Session auf dem simulierten \textit{iPhone} ohne \textit{XCode} hergestellt werden kann.\\

\begin{figure}[htbp]
	\centering
	\includegraphics[width=\textwidth]{bilder/pentest_mobile_anwendungen/vergleich_aktuelle_situation/20160627_lsof_XCode_running.png}
	\caption{Vergleich der offenen Pipes vor und nach der Ausführung der Applikation in XCode}
	\label{fig:LSOFLLDB}
\end{figure}

Nach einem Artikel von Apple\footnote{\url{https://developer.apple.com/library/ios/documentation/IDEs/Conceptual/gdb_to_lldb_transition_guide/document/lldb-terminal-workflow-tutorial.html}} ist es möglich, mit \textit{LLDB} eine App auch als "`Standalone Debugger"', also ohne \textit{XCode}, zu verwenden. Dies ist in Abbildung \ref{fig:LLDBStandaloneDebugger} aufgezeigt.\\

\begin{figure}[htbp]
	\centering
	\includegraphics[width=\textwidth]{bilder/pentest_mobile_anwendungen/vergleich_aktuelle_situation/20160627_LLDB-Standalone-Debugger.png}
	\caption{LLDB als Standalone Debugger}
	\label{fig:LLDBStandaloneDebugger}
\end{figure}

Um zu verifizieren, dass die App auf einem simulierten \textit{iPhone} ausgeführt wird, können entweder die geöffneten Prozesse (siehe Abbildung \ref{fig:LLDB-creating-IPhone-VM}) oder die geladenen Bibliotheken der Programme (siehe Abbildung \ref{fig:VergleichLLDBImages}) verglichen werden.\\

\begin{figure}[htbp]
	\centering
	\includegraphics[width=\textwidth]{bilder/pentest_mobile_anwendungen/vergleich_aktuelle_situation/20160627_LLDB-image-list.png}
	\caption{Vergleich der geladenen Bibliotheken}
	\label{fig:VergleichLLDBImages}
\end{figure}

Beide Methoden zeigen, dass die App auf dem simulierten \textit{iPhone} gestartet wird. Bei den Prozessen ist zu beobachten, dass vor Start von \textit{LLDB} nur der Hintergrund-Service ausgeführt wurde. Nach dem Start von \textit{LLDB} dagegen läuft der gesamte Simulator.\\

Auch der Vergleich der geladenen Bibliotheken legt nahe, dass die \textit{LLDB} Standalone und \textit{Xcode} in der gleichen Umgebung ausgeführt werden. Die Adressen im RAM variieren aufgrund von \textit{ASLR} zwar, aber es werden dieselben Bibliotheken verwendet (am Pfad zu erkennen). Dies ist in Grafik \ref{fig:VergleichLLDBImages} dargestellt.

\begin{figure}[htbp]
\lstinputlisting[lastline=16]{logs/20160627_LLDB-creating-IPhone-VM.txt}
\caption{Geöffnete Prozesse vor und während der Ausführung des Emulators}
\label{fig:LLDB-creating-IPhone-VM}
\end{figure}

\subsubsection{Architektur}\label{ref:VergAktSitiOSArch}
Auffällig ist, dass der Simulator nicht die CPU-Architektur eines \textit{iPhones} simuliert, sondern den Code auf dem x86-Prozessor des Mac-Books ausführt. Dies hat den Nachteil, dass Apps aus dem Apple-App-Store, welche für ARM-Prozessoren kompiliert sind, nicht auf dem Simulator ausgeführt werden können. Lediglich Apps, welche für den Simulator in \textit{Xcode} kompiliert wurden, können im Simulator ausgeführt werden. Dadurch ist eine Analyse von Apps ohne Quellcode-Zugriff nicht möglich.

\subsubsection{Sicherheits-Aspekte}
Bei der Analyse von iOS-Apps wurden zwei mögliche Sicherheitslücken testweise in eine App implementiert. Die Lücken sind "`unsichere Funktionen"', welche eine eventuelle \textit{Memory Corruption} mit sich ziehen, und Verbindungen ohne \textit{TLS}-Absicherung.

\pagebreak
\paragraph{Unsichere Funktionen}
In \textit{Objective C} für \textit{iOS}-Apps sind Funktionen, welche für \textit{Memory Corruption}-Schwachstellen bekannt sind, leider noch vorhanden.\\

So führt folgendes Code-Segment zwar zum Absturz der App, aber könnte bei einer dynamischen Eingabe des zu kopierenden Strings durchaus eine echte Schwachstelle einführen.
\begin{lstlisting}
char msg[15];
char *str = "\x41\x41\x41\x41\x41\x41\x41\x41\x41\x41\x41\x41\x41\x41\x42";
strcpy(msg, str);
\end{lstlisting}

Dies ist auch in der Apple-Online-Dokumentation beschrieben\footnote{\url{https://developer.apple.com/library/content/documentation/Security/Conceptual/SecureCodingGuide/Articles/BufferOverflows.html}}.

\paragraph{Ungesicherte Verbindungen}\label{ref:inseccon}
Ein Sicherheitsfeature ab \textit{iOS 9.0} ist der Umgang von \textit{iOS} mit Netzwerk-Verbindungen. So können ohne erweiterte Konfiguration keine Verbindungen aufgebaut werden, welche nicht dem RFC-Standard 2818\footnote{\url{https://tools.ietf.org/html/rfc2818}} folgen. So führt der in Listing \ref{lst:VergliOSTLSAufbau} gezeigte Aufruf einer HTTP-Seite zu der in \ref{lst:VergliOSTLSFehler} gezeigten Fehlermeldung.\\

\begin{figure}[p]
\begin{lstlisting}
NSURL *url = [NSURL URLWithString:@"http://api.ipify.org"];
    NSData *data = [NSData dataWithContentsOfURL:url];
    NSString *ret = [[NSString alloc] initWithData:data encoding:NSUTF8StringEncoding];
    NSLog(@"ret=%@", ret);
\end{lstlisting} 
\caption{Aufbau einer nicht mit TLS gesicherten Verbindung}
\label{lst:VergliOSTLSAufbau}
\end{figure}

\begin{figure}[p]
\begin{lstlisting}
2016-06-28 08:42:56.518 Test2[4789:140270] App Transport Security has blocked a cleartext HTTP (http://) resource load since it is insecure. Temporary exceptions can be configured via your app's Info.plist file.
\end{lstlisting}
\caption{Fehler bei Aufbau einer ungesicherten Verbindung ohne Definition einer Ausnahme}
\label{lst:VergliOSTLSFehler}
\end{figure}

Sollten unsichere Verbindungen benötigt werden, muss ein entsprechender Eintrag in der "`Info.plist"' angelegt werden. Dieser Eintrag muss sehr genau auf die App angepasst werden, da zum Beispiel die Domains festgelegt werden müssen. Ein Eintrag muss laut Apple\cite{AppleNSAppTransportSecurity} den in \ref{lst:VergliOSATSAus} dargestellten Inhalt haben. Eine Überprüfung auf solche Ausnahmen kann dem Abschnitt \ref{ref:WeitMobSFErkennungVonUngesichertenVerbindungen} entnommen werden.

\begin{figure}
\begin{lstlisting}
NSAppTransportSecurity : Dictionary {
    NSAllowsArbitraryLoads : Boolean
    NSAllowsArbitraryLoadsForMedia : Boolean
    NSAllowsArbitraryLoadsInWebContent : Boolean
    NSAllowsLocalNetworking : Boolean
    NSExceptionDomains : Dictionary {
        <domain-name-string> : Dictionary {
            NSIncludesSubdomains : Boolean
            NSExceptionAllowsInsecureHTTPLoads : Boolean
            NSExceptionMinimumTLSVersion : String
            NSExceptionRequiresForwardSecrecy : Boolean   // Default value is YES
            NSRequiresCertificateTransparency : Boolean
        }
    }
}
\end{lstlisting}
\caption{XML-Definition eines einer Ausnahme der ATS}
\label{lst:VergliOSATSAus}
\end{figure}

		\subsection{Windows-Phone}
		Im Folgenden wurde der in \textit{Visual Studio} enthaltene Emulator für \textit{Windows-Phone}-Apps in seiner Funktionsweise untersucht.
		
			\subsubsection{Emulation}
			Zur Emulation von \textit{Windows-Phones} ist \textit{Visual Studio} notwendig. Im Folgenden wurde \textit{Visual Studio 15} in der \textit{Community Edition}, also einer kostenlosen Version, verwendet.\\
			
			Nach der Installation von \textit{Visual Studio} sollte das Framework für die \textit{Universal Windows Platform} sowie der Emulator installiert werden. Beide können als Optionen bei der Installation gewählt werden. Anschließend kann eine App sowohl im Emulator, als auch lokal auf der Windows-Maschine gestartet werden. Alternativ kann auch ein physikalisches Windows-Gerät für die Ausführung von Apps genutzt werden.
			
			\subsubsection{Debugging}
			Debugging kann direkt über \textit{Visual Studio} durchgeführt werden. Dazu können einfach die jeweiligen Breakpoints im Code gesetzt werden.\\
			
			Eine Möglichkeit zum Debugging außerhalb von \textit{Visual Studio} wurde nicht gefunden, da Microsoft keine öffentlich beschriebene Schnittstelle zur Kommunikation mit Apps innerhalb des Betriebssystems zur Verfügung stellt.
			
			\subsubsection{Unsichere Funktionen}
			Unsichere Funktionen wie "`strcpy"' oder "`memcpy"' sind in den Bibliotheken für \textit{C++}-Apps der \textit{Universal Windows Platform} noch enthalten, können aber ohne erweitere Konfiguration nicht genutzt werden.\\
			
			Um diese Funktionen zu nutzten, muss dem Compiler die Flag 
			\begin{lstlisting}
_CRT_SECURE_NO_WARNINGS			
			\end{lstlisting}
			übergeben werden. Anschließend können auch Funktionen, welche zu \textit{Memory-Corruption} führen können, frei genutzt werden.
			
			\subsubsection{Unsichere Verbindungen}
			Bezüglich unsicherer Verbindungen schreibt die \textit{Universal Windows Platform} nichts vor. So kann über folgenden Aufruf eine Seite ohne TLS aufgerufen werden.
			\begin{lstlisting}
auto uri = ref new Windows::Foundation::Uri("http://api.ipify.org");
webView->Navigate(uri);
			\end{lstlisting}
			
		\subsection{Android}
			\textit{Android} ist ein ursprünglich 2003 von der \textit{Android, Inc.} entwickeltes mobiles Betriebssystem. 2005 wurde es durch Google übernommen und wird seitdem weiterentwickelt. 2015 liegt es in der EU bei einem Marktanteil von $75,6\%$\cite{KatarWorldpanelMobBetrSys}. Aufgrund der Quelloffenheit des Systems wird es von vielen Herstellern auf verschiedensten Plattformen genutzt. Jedoch bringt die weitführende Fragmentierung des Betriebssystems auch Nachteile mit sich. So sind im Februar 2017 nur $1,2 \%$ der Android-Devices auf einer aktuellen Version (Nougat).\cite{StatistaAndroidVersionen}\cite{Drake2014}
			
			\subsubsection{Android-Studio und SDK}
			Das Android-Studio ist eine umfassende IDE. Sie ermöglicht unter anderem das schnelle Entwickeln und Testen von Apps, sowie die Emulation von beliebigen Android-Versionen. Außerdem ist Android Studio kostenlos, Open-Source und für Linux, Mac und Windows erhältlich. Die aktuelle Version kann von der offiziellen Webseite\footnote{\url{http://developer.android.com/sdk/index.html}} heruntergeladen werden. Die Installation unter Linux ist vergleichsweise einfach, da nur ein Archiv über das Kommando 
\begin{lstlisting}
unzip android-studio-ide-143.2739321-linux.zip
\end{lstlisting}
entpackt werden muss. Für alle anderen Betriebssysteme werden entsprechende Installationsroutinen zur Verfügung gestellt. Anschließend kann die IDE über die Datei "`bin/studio.sh"' gestartet werden. Neben dem Android-Studio gibt es noch das Android SDK, welches über die gleiche URL heruntergeladen werde kann. Es enthält wichtige Kommandozeilen-Tools wie \textit{adb}, \textit{fastboot} oder \textit{logcat}, auf welche zum Teil im weiteren Verlauf noch detailliert eingegangen wird.

			%\subsubsection{Compatibility Testing Suite}
			%\cite{Drake2014} Seite 18
			
			\subsubsection{Emulation vs. Hardware}
			Im Gegensatz zum \textit{iOS}-Simulator hat \textit{AVD} die Möglichkeit, CPU wie auch GPU eines Handys zu emulieren. Dabei besteht die Auswahl zwischen verschiedenen Architekturen, wie Intel x86 oder ARM. Zudem gibt es viele andere Möglichkeiten zur Konfiguration der einzelnen Maschinen, wie in Grafik \ref{fig:VergleichAVDConfig} zu sehen ist. Daher hat \textit{Android} den Vorteil, dass gerade tiefgreifende Analysen aufgrund der Emulation der Architektur näher an der echten Hardware sind als bei \textit{iOS}.\cite{Drake2014}
			
			\begin{figure}[htbp]
				\centering
				\includegraphics[width=\textwidth]{bilder/pentest_mobile_anwendungen/vergleich_aktuelle_situation/20170215_AVD-Config-Screen.png}
				\caption{Der Konfigurations-Bildschirm des AVD-Managers}
				\label{fig:VergleichAVDConfig}
			\end{figure}
			
			\subsubsection{Debugging}	
			Zum Debugging unter Android kann die \textit{Android Debug Bridge}, kurz \textit{ADB}, genutzt werden. Dabei bietet \textit{ADB} nicht nur die Funktionen für emulierte Geräte, sondern auch für physikalische. Zusätzlich kann für physikalische Geräte sogar Debugging über das Netzwerk durchgeführt werden. Für kabelgebundene oder emulierte Handys können mit dem Aufruf \textit{adb devices} die dem Rechner zur Verfügung stehenden Geräte abgerufen werden. Anschließend können über \textit{adb shell} beliebige Kommandos, inklusive des Aufrufs des internen Debuggers, gegeben werden.\cite{androidDebugBridge}\\
			
			Eine App kann über das Kommando
			\begin{lstlisting}
am start -n <package identifier>/.<activity>
			\end{lstlisting}
			gestartet werden. Dazu muss jedoch die Start-Aktivität der App in Erfahrung gebracht werden. Dies ist über
			\begin{lstlisting}
cmd package resolve-activity --brief <package identifier>
			\end{lstlisting}
			möglich. Um eine App im Debug-Modus zu starten, wird \textit{am} einfach der Parameter \textit{-D} angehängt.
			\begin{lstlisting}
am start -n -D <package identifier>/.<activity>
			\end{lstlisting}
			Allerdings bleibt anzumerken, dass die Debug-Flag für die meisten Apps nicht gesetzt ist. Um dies, insofern notwendig, zu umgehen, muss \textit{am}	als privilegierter User ausgeführt werden. Damit dies möglich ist, muss das Smartphone "`gerooted"', also der Root-Account aktiviert, sein. Alternativ kann auch das APK vom Telefon geladen, die Debug-Flag im \textit{AndroidManifest.xml} ergänzen und wieder am Handy installiert werden.\\
			
			Anschließend kann die App direkt über \textit{jdb}\footnote{\url{https://docs.oracle.com/javase/8/docs/technotes/tools/windows/jdb.html}} oder eine IDE wie IntelliJ gedebugged werden. Im Detail ist dies zum Beispiel im Blog\footnote{\url{https://blog.netspi.com/attacking-android-applications-with-debuggers/}} von Eric Gruber beschreiben.\\
			
%			Abschließend bleibt zu sagen, dass das Debugging von \textit{Android}-Apps über eine Vielzahl von Wegen und Tools durchgeführt werden kann.
			
			%\subsubsection{Logcat}
			
			%monitor
			%Android Debug Bridge\cite{androidDebugBridge}

\newpage
\section{Abgleich der Anforderungen mit MobSF}\label{ref:PenMobAnwAbgAndMobSF}

Im folgenden Abschnitt werden die Anforderungen aus \ref{ref:PenMobAnwdWeiterAnford} mit den bestehenden Eigenschaften von \textit{MobSF} abgeglichen.

\begin{description}
	\item[Einfache Einrichtung] \textit{MobSF} benötigt keine echte Installation, sondern kann einfach aus dem Github-Repository heruntergeladen werden. Die anschließenden Konfigurationsschritte unterscheiden sich leicht nach Betriebssystem, bestehen jedoch grundsätzlich immer aus der Installation von Python sowie der Installation der Abhängigkeiten über das Python-Tool \textit{pip}. Abschließend sollten die Einstellungen an die Umgebung angepasst werden. Damit ist die Installation nicht trivial, aber gut dokumentiert und durch eine Person mit Fachkenntnis in kurzer Zeit durchzuführen.
	
	\item[Einfache Handhabung] Die Handhabung von \textit{MobSF} wird über eine HTML-Oberfläche abgebildet. Samples können per Drag-n-Drop zur Analyse eingereicht werden. Die Realisierung als Web-Anwendung hat den Vorteil, dass das Tool zentral installiert und von verschiedenen Orten über das Netzwerk genutzt werden kann.
	
	\item[Verständliche Ergebnisse] Die Ergebnisse einer Analyse werden ebenfalls in der HTML-Anwendung dargestellt. Dabei ist über Farb-Codes einfach zu erkennen, ob Probleme mit der App bestehen. Eine genaue Analyse der Ergebnisse und weiterführende Untersuchungen sollten jedoch durch Experten durchgeführt werden.
	
	\item[Unterstützung für Android/iOS/Windows Phone] Vor der Weiterentwicklung von \textit{MobSF} unterstützte das Framework die statische und dynamische Analyse von \textit{Android}-Apps sowie die statische Analyse von \textit{iOS}-Apps. Eine Analyse von \textit{Windows-Phone}-Apps ist noch nicht enthalten.
	
	\item[Niedrige False-Positive-Rate] \textit{MobSF} beschreibt nicht direkt Schwachstellen, sondern Indikatoren wie beispielsweise die Verwendung von verwundbaren Funktionen. Daher sind die gegebenen Warnungen stets korrekt, jedoch wird womöglich auch in manchen Bereichen nicht alles entdeckt.
	
	\item[Reproduzierbarkeit] \textit{MobSF} benutzt zumeist Tools, welche in \textit{MobSF} selbst enthalten sind. Daher sollte dieselbe Version von \textit{MobSF} auch auf verschiedenen Rechnern gleiche Ergebnisse liefern.
	
\end{description}

Somit sind alle Anforderungen bis auf die Unterstützung von Windows-Phone-Apps ausreichend erfüllt.

\pagebreak	
\section{Laboraufbau}
Zur Weiterentwicklung des \textit{MobSF} wurde unterschiedliche Hardware mit verschiedenen Werkzeugen genutzt.\\

Als Hardware wurde durch die Allianz Deutschland AG ein Apple Mac-Book Pro mit ausreichenden Ressourcen gestellt (i7, 16GB RAM). Als Betriebssystem wurde anfangs \textit{Mac OS X Maverick} und später \textit{Sierra} verwendet.\\

Um die Entwicklung für Windows-spezifische Anwendungen zu realisieren, wurde eine virtuelle Maschine mit \textit{Windows  10} über \textit{VMWare Fusion}\footnote{\url{https://www.vmware.com/de/products/fusion.html}} verwendet.\\

Als Software wurde Python\footnote{\url{https://www.python.org/}} in Version 2.7 und 3.6 sowie die jeweiligen Abhängigkeiten in Form von Modulen genutzt.\\

Als Entwicklungsumgebung wurde \textit{Atom}\footnote{\url{https://atom.io/}} mit \textit{Pylint}\footnote{\url{https://www.pylint.org/}} für Python 2.7/3.6 und \textit{VIM}\footnote{\url{http://www.vim.org/}} genutzt.

%\section{Entwicklung der Umgebung}
%	\subsection{Aufbau}
	

%	\subsection{Schnittstellen}

\section{Weiterentwicklung MobSF}
\label{Weiterentwicklung MobSF}
Ein Kernelement dieser Arbeit ist die Weiterentwicklung des Mobile Security Frameworks \textit{MobSF}. Die Funktionen des Frameworks sind bereits unter Abschnitt \ref{ref:PenMobAnwAbgAndMobSF} aufgezeigt. Im Folgenden sind die Änderungen dargestellt, welche an dem Framework vorgenommen und veröffentlicht wurden.

\subsection{Allgemeine Verbesserungen}
Neben Verbesserungen, welche einem genauen Bereich (\textit{Windows-Phone}, \textit{iOS}, \textit{Android}) zuzuordnen sind, gibt es auch einige allgemeine Erweiterungen am \textit{MobSF}. Diese sind ebenfalls im Folgenden dargestellt.

\subsubsection{Struktur}
Die Struktur von \textit{MobSF} war bisher relativ flach. Auf der ersten Ebene findet man die übergeordneten Module wie den \textit{ApiTester}, \textit{StaticAnalyzer}, \textit{DynamicAnalyzer} sowie den \textit{statischen Content}, \textit{Templates} und Kern-Module des \textit{MobSF}. Dies ist in der Abbildung \ref{fig:MobSFStrukOrig} verdeutlicht. Jedoch hatte die Struktur in den Modulen oft keine saubere Trennung der Aufgaben. So waren im \textit{StaticAnalyzer}-Modul sowohl \textit{iOS} wie auch \textit{Android}-Analyse in der \textit{views.py} zusammengefasst.\\

\begin{figure}[htbp]
\dirtree{%
.1 Mobile-Security-Framework-MobSF/.
    .2 .git/.
    .2 APITester/.
    .2 downloads/.
    .2 DynamicAnalyzer/.
    .2 LICENSES/.
    .2 logs/.
    .2 MalwareAnalyzer/.
    .2 MobSF/.
    .2 static/.
    .2 StaticAnalyzer/.
    .2 templates/.
    .2 uploads/.
}
\caption{Struktur des MobSF auf der ersten Ebene}
\label{fig:MobSFStrukOrig}
\end{figure}

\pagebreak
 Um hier eine klarere Trennung zu schaffen, wurde die \textit{views.py} in einem ersten Schritt aufgegliedert in drei Dateien:
\begin{description}
	\item[shared\_func.py: ] Die \textit{shared\_func.py} enthält alle Funktionen, welche sowohl für \textit{iOS} als auch \textit{Android} gebraucht werden. Beispiele sind die Erstellung von Hashes, das Generieren von PDFs oder das Entpacken von Archiven.
	
	\item[ios.py: ] Die Datei \textit{ios.py} enthält alle \textit{iOS} spezifischen Funktionen zur statischen Analyse.
	
	\item[android.py: ] Die Datei \textit{android.py} enthält alle \textit{Android} spezifischen Funktionen zur statischen Analyse.
	
	\item[windows.py: ] Die Datei \textit{windows.py} enthält alle \textit{Windows-Phone} spezifischen Funktionen zur statischen Analyse. Sie wurde nachträglich hinzugefügt (siehe \ref{Windows-Apps}), weshalb die Funktionen in der alten Struktur nicht auftauchen.
\end{description}
Dies ist im Detail in der Abbildung \ref{fig:MobSFStaticStrucVergl} dargestellt.\\

\begin{figure}
\centering
\begin{subfigure}{.5\textwidth}
  \dirtree{%
.1 StaticAnalyzer/.
    .2 [...].
    .2 views.py.
    	.3 key.
    	.3 PDF.
    	.3 Java.
    	.3 Smali.
    	.3 Find.
    	.3 ViewSource.
    	.3 ManifestView.
    	.3 StaticAnalyzer.
    	.3 GetHardcodedCertKeystore.
    	.3 ReadManifest.
    	.3 GetManifest.
    	.3 ValidAndroidZip.
    	.3 HashGen.
    	.3 FileSize.
    	.3 GenDownloads.
    	.3 zipdir.
    	.3 Unzip.
    	.3 FormatPermissions.
    	.3 CertInfo.
    	.3 WinFixJava.
    	.3 WinFixPython3.
    	.3 Dex2Jar.
    	.3 Dex2Smali.
    	.3 Jar2Java.
    	.3 Strings.
    	.3 ManifestData.
    	.3 ManifestAnalysis.
    	.3 CodeAnalysis.
    	.3 StaticAnalyzer\_iOS.
    	.3 ViewFile.
    	.3 readBinXML.
    	.3 HandleSqlite.
    	.3 iOS\_ListFiles.
    	.3 BinaryAnalysis.
    	.3 iOS\_Source\_Analysis.
}
  \caption{Alte Struktur}
  \label{fig:sub1}
\end{subfigure}%
\begin{subfigure}{.5\textwidth}
  \dirtree{%
.1 StaticAnalyzer/.
    .2 [...].
    .2 views/.
		.3 android.py.
	    	.4 [...].
	    	.4 GetHardcodedCertKeystore.
	    	.4 ReadManifest.
	    	.4 GetManifest.
	    	.4 ValidAndroidZip.
	    	.4 Dex2Jar.
	    	.4 Dex2Smali.
	    	.4 Jar2Java.
	    	.4 [...].
		.3 ios.py.
	    	.4 StaticAnalyzer\_iOS.
	    	.4 ViewFile.
	    	.4 readBinXML.
	    	.4 HandleSqlite.
	    	.4 iOS\_ListFiles.
	    	.4 BinaryAnalysis.
	    	.4 iOS\_Source\_Analysis.
	    .3 windows.py.
	    	.4 [...].
	    	.4 \_\_binskim.
	    	.4 \_\_binscope.
	    	.4 [...].
		.3 shared\_func.py.
	    	.4 key.
	    	.4 FileSize.
	    	.4 HashGen.
	    	.4 Unzip.
	    	.4 PDF.
}
  \caption{Neue Struktur}
  \label{fig:sub2}
\end{subfigure}
\caption{Vergleich der Struktur von \textit{StaticAnalyzer}}
\label{fig:MobSFStaticStrucVergl}
\end{figure}

Im weiteren Verlauf der Weiterentwicklung und mit der Einführung von Code-Standards (siehe \ref{pylintering}) wurde die Struktur weiter verfeinert. So wurden die Datei "`android.py"' entsprechend der Funktionalitäten weiter aufgeteilt. So ergibt sich für die statische Analyse von Android-Apps die in \ref{fig:MobSFStrukAndroid} dargestellte Dateistruktur.

\begin{figure}
\dirtree{%
.1 Mobile-Security-Framework-MobSF/.
	.2 [...].
    .2 StaticAnalyzer/.
    	.3 [...].
    	.3 views/.
            .4 \_\_init\_\_.py.
            .4 ios.py.
            .4 shared\_func.py.
            .4 windows.py.
            .4 android/.
                .5 \_\_init\_\_.py.
                .5 binary\_analysis.py.
                .5 cert\_analysis.py.
                .5 code\_analysis.py.
                .5 converter.py.
                .5 db\_interaction.py.
                .5 dvm\_permissions.py.
                .5 find.py.
                .5 java.py.
                .5 manifest\_analysis.py.
                .5 manifest\_view.py.
                .5 smali.py.
                .5 static\_analyzer.py.
                .5 strings.py.
                .5 view\_source.py.
                .5 win\_fixes.py.
}
\caption{Struktur der Dateien für die statische Analyse}
\label{fig:MobSFStrukAndroid}
\end{figure}

\newpage
\subsubsection{Code-Standards}\label{pylintering}
Um eine hohe Code-Qualität zu gewährleisten, sollte homogener und gut wartbarer Code geschrieben werden. Um dies über mehrere Entwickler hinweg zu gewährleisten, gibt es sogenannte \textit{Code-Standards}. Diese werden meist von einer zentralen Stelle festgelegt und beschreiben, zum Beispiel wie Variablen benannt werden müssen, wie viele Kommentare nötig sind und wie lang eine Funktion maximal sein darf.\\

Bei der Neu- oder Reimplementierung wurde auf die Verwendung von offiziellen \textit{Code-Standards} geachtet. Insbesondere wurde der\textit{ PEP 8} Standard \footnote{\url{https://www.python.org/dev/peps/pep-0008/}} für \textit{Python} verwendet, welcher die Lesbarkeit und Wartbarkeit von \textit{Python}-Code verbessern soll. Um die Einhaltung des Standards zu gewährleisten, wurde das Tool \textit{Pylint} verwendet. Dieses prüft einen gegebenen Quellcode gegen den Code-Standard \textit{PEP 8} und kreiert entsprechende Warnungen für Abweichungen. Ursprünglich musste \textit{Pyling} auf der Konsole extra ausgeführt werden, jedoch können in moderne Entwicklungsumgebungen wie \textit{Atom} Tool wie \textit{Pylint} direkt eingebunden und genutzt werden. Dies hat den Vorteil, dass bereits während des Programmierens Verstöße gegen Standards oder Fehler wie zum Beispiel falsche Variablennamen entdeckt werden.\\

Für \textit{MobSF} wurde nicht von Anfang an mit Code-Standards entwickelt, weshalb zum Beispiel die Datei \textit{android.py} auf 2000 Code-Zeilen über 1600 \textit{Pylint}-Fehler aufwies. Durch aufwendige Refaktorierungs- und Umstrukturierungsarbeiten konnte die Anzahl der Abweichungen massiv reduziert werden.\\

Daraus ergibt sich erhöhte Wartbarkeit sowie eine einfachere Weiterentwicklung aufgrund weniger Merge-Konflikte. Dies ist möglich, da die jeweiligen Methoden je nach Funktionalität in entsprechende Module ausgelagert wurden und somit nur die eine, für die Funktion benötigte Datei, verändert und wieder in das Haupt-Projekt eingegliedert werden muss. Zuvor musste bei paralleler Entwicklung am Projekt jede auch noch so kleine Änderung in der übergreifenden Datei mit Änderungen einer parallel arbeitenden Partei zusammengeführt werden, was oftmals viel Arbeit bedeutet.

\subsubsection{strings}\label{ref:strings}
Zuerst wurde das \textit{MobSF} um die Fähigkeit erweitert, eine \textit{iOS}-Applikation mit dem \textit{strings}-Programm zu untersuchen. \textit{strings} durchsucht, sofern keine zusätzlichen Parameter übergeben werden, eine binäre Datei auf \textit{ASCII}-Elemente mit mindestens vier Stellen und gibt diese anschließend zurück.\\

Dies hilft oft bei einer ersten Einschätzung der Anwendung, da zumeist eine grundlegende Funktionsweise und der Zweck der Software abgeleitet werden können. Ebenso können eventuell unbeabsichtigt im Programm vergessene oder eigentlich geheime Strings in einer App aufgedeckt werden, wie zum Beispiel Entwicklerkommentare oder Passwörter. Auch sind je nach Compiler-Einstellungen Funktionsnamen oder Calls in andere Bereiche (z.B. Import-Table bei PE-Files) als String in einem Binary enthalten, was unter Abschnitt \ref{ref:WeitEntwWinBadFunc} zum Entdecken von verwundbaren Funktionen genutzt wird.\\

Sowohl \textit{Mac OS X} wie auch \textit{Linux} haben ein integriertes \textit{string}-Kommando, welches jedoch \textit{Windows} fehlt. Um die Multi-Platform-Fähigkeit weiterhin zu gewährleisten, wurde die Funktion in Python-Code abgebildet. Als Vorlage wurde ein bestehender Code von \textit{Stackoverflow} \footnote{\url{http://stackoverflow.com/a/17197027}} genutzt. Dieser lieferte jedoch eine wesentlich höhere Anzahl von Ergebnissen, da bestimmte Whitespace-Character ebenfalls beachtet wurden (siehe Abbildung \ref{lst:WeitMobSFVerglSrings}).\\

\begin{figure}[htbp]
\begin{lstlisting}
> wc -l strings_test_*
	  85149 strings_test_orig
  	541393 strings_test_pyth
    626542 total
\end{lstlisting}
\caption{Vergleich der gefundenen Strings mit Python und dem ursprünglichen Strings-Kommando}
\label{lst:WeitMobSFVerglSrings}
\end{figure}

Da viele der zusätzlich aufgedeckten Strings jedoch nicht bei der Analyse geholfen, sondern eher das Auffinden relevanter Strings erschwert haben, wurde der Originalcode wie in Abbildung \ref{lst:WeitMobSFStringsAnpassungen} dargestellt angepasst. Durch die Reduzierung der ausschlaggebenden Zeichen konnten die Ergebnisse optimiert werden, sodass eine effiziente Suche über Strings wieder möglich ist.

\begin{figure}[htbp]
\begin{lstlisting}[escapechar=\ü]
<<<<<<< Vor Anpassung
import string
=======
>>>>>>> Nach Anpassung

def strings(filename, min=4):
    """Print out all connected series of readable chars longer than min."""
    with open(filename, "rb") as f:
        result = ""
        for c in f.read():
<<<<<<< Vor Anpassung
            if c in string.printable:
=======
            if c in (
                    '0123456789'
                    'abcdefghijklmnopqrstuvwxyz'
                    'ABCDEFGHIJKLMNOPQRSTUVWXYZ'
                    '!ü"\#ü$%&\'()*+,-./:;<=>?@[\\]^_`{|}~ '
            ):
>>>>>>> Nach Anpassung
                result += c
                continue
            if len(result) >= min and result[0].isalnum():
                yield "'" + result + "'"
            result = ""
\end{lstlisting}
\caption{Anpassungen am Code der Python-Implementierung von Strings zur Verbesserung der Ergebnisse}
\label{lst:WeitMobSFStringsAnpassungen}
\end{figure}

\subsubsection{PDF-Generation}
Um eine effiziente Weitergabe der Ergebnisse zu ermöglichen, besitzt \textit{MobSF} eine PDF-Export-Funktion. Die Implementierung ist dabei relativ einfach. Es wird eine neue HTML-Sicht geschaffen, welche anschließend über das Python-Modul \textit{xhtml2pdf.pisa} als PDF geöffnet wird. Aufgrund von Leistungsproblemen des Moduls wurde später auf \textit{pdfkit}\footnote{\url{https://pypi.python.org/pypi/pdfkit}} gewechselt.\\

Eine solche Sicht wurde jeweils für alle Erweiterungen implementiert, welche in dieser Arbeit vorgenommen wurden.

\newpage
\subsubsection{RPC-Service}\label{ref:XMLRPC}
Um eine Kommunikation zwischen verschiedenen virtuellen Maschinen zu ermöglichen, wurde ein minimaler \textit{RPC-Server} in Python 3.5 entwickelt. Hierzu wurden verschiedene Ansätze untersucht. Getestet wurden hierzu die Python-Module \textit{Flask}, \textit{Requests}, \textit{xmlrpc} sowie \textit{RSA} zum Hinzufügen einer Authentisierung.

\paragraph{Flask}\label{ref:flask}
Flask ist ein schneller, minimaler Webserver. Mit nur sehr wenig Code es möglich, eine Schnittstelle bereit zu stellen. Ein Code mit zwei akzeptierenden Funktionen, basierend auf der Schnellstart-Anleitung\footnote{\url{http://flask.pocoo.org/docs/0.10/quickstart/}}, ist in Abbildung \ref{ref:rpc_client.py} dargestellt.\\

\begin{figure}
\begin{lstlisting}
from flask import Flask
app = Flask(__name__)

@app.route('/')
def hello_world():
	print("Execute command!")
    return 'Hello World!'
    
@app.route('/second_command/')
def not_hello_world():
	print("Execute second command!")
    return 'Goodbye World!'

if __name__ == '__main__':
    app.run()
\end{lstlisting}
\caption{rpc\_client.py}
\label{ref:rpc_client.py}
\end{figure}

Es wird eine minimale Anwendung erstellt, welche auf dem Pfad "`\url{/}"' lokal das \textit{print}-Statement ausführt und den Text "`Hello World!"' zurück gibt. Wird die Anwendung unter dem Pfad "`\url{/second_command/}"' angesprochen, wird ein anderes \textit{print}-Statement ausgeführt und ein anderer Wert zurück gegeben. Auf diese Weise können schnell API-Funktionen auf verschiedene Pfade gelegt und angesprochen werden.

\newpage
\paragraph{Requests}
\textit{Requests} ist ein Python-Modul, welches einfache Anfragen (sogenannte \textit{Requests}) über HTTP(S) ermöglicht. So ist es über ein kurzes Code-Snippet, dargestellt in Abbildung \ref{ref:rpc_server.py}, möglich, die unter \ref{ref:rpc_client.py} aufgezeigt Schnittstelle anzusprechen.\\

\begin{figure}
\begin{lstlisting}
import requests

r = requests.get('http://localhost:5000')
print(r.text)

r = requests.get('http://localhost:5000/second_command/')
print(r.text)
\end{lstlisting}
\caption{rpc\_server.py}
\label{ref:rpc_server.py}
\end{figure}

Wird zuerst der Code \ref{ref:rpc_client.py} und anschließend der Code \ref{ref:rpc_server.py} ausgeführt, wird auf Server-Seite die in Abbildung \ref{lst:WeitMobSFRPCAusgClient} gezeigte Ausgabe erzeugt.\\
\begin{figure}[h]
\begin{lstlisting}
 $ python3 rpc_server.py
Hello World!
Goodbye World!
\end{lstlisting}
\caption{Ausgabe des RPC-Servers}
\label{lst:WeitMobSFRPCAusgServ}
\end{figure}

Auf der Client-Seite erfolgt die in Abbildung \ref{lst:WeitMobSFRPCAusgServ} dargestellte Ausgabe.
\begin{figure}[h]
\begin{lstlisting}
 $ python3 rpc_client.py
 * Running on http://127.0.0.1:5000/
Execute command!
127.0.0.1 - - [18/May/2016 19:10:56] "GET / HTTP/1.1" 200 -
Execute second command!
127.0.0.1 - - [18/May/2016 19:10:56] "GET /second_command/ HTTP/1.1" 200 -
\end{lstlisting}
\caption{Ausgabe des RPC-Clients}
\label{lst:WeitMobSFRPCAusgClient}
\end{figure}

Auf dieser Basis wurde die erste Version des RPC-Servers implementiert.

\newpage
\paragraph{xmlrpc mit rsa}
Bei der Eingliederung der Verbindung zwischen der virtuellen Maschine und dem Host wurde die Anforderung nach einer Möglichkeit zur Authentifizierung gefordert.\\

Zur Implementierung dieser Anforderung wurde das Python-Modul \textit{rsa} verwendet, welches sowohl für \textit{Python} 2 wie auch \textit{Python} 3 existiert. Die Ablauf der Kommunikation ist im Diagramm \ref{fig:WeitMobSFXMLRPCAblauf} dargestellt.\\

\tikzset{
    state/.style={
           rectangle,
           rounded corners,
           draw=black, very thick,
           minimum height=2em,
           inner sep=2pt,
           text centered,
           },
    bigbox/.style={
    		draw, 
    		inner sep=20pt,
    		label={[shift={(0ex,0ex)}]:#1}},
}

\begin{figure}[p]


\begin{tikzpicture}[node distance=4cm]

\node[state,initial]             (start) {Wartezustand};
\node[state,below=of start]             (an_start) {Analyse anfordern};
\node[state,right=of an_start]       (chal_calc) {Challenge berechnen};
\node[state,below =of an_start] (chal_sig) {Challenge signieren};
\node[state,below=of chal_calc] (check_chal) {Prüfen von Sig. und Chal.};
\node[state,below=of check_chal] (analyse) {Durchführung der Analyse};
\node[state,below=of chal_sig] (dar_erg) {Darstellung der Ergebnisse};

\node[bigbox=MobSF Host, label={[shift={(15ex,3ex)}]}, fit=(start)(an_start)(chal_sig)(dar_erg),] (MobSFHost) {};
\node[bigbox=Analyse Server, fit=(chal_calc)(check_chal)(analyse),] (MobSFHost) {};

\draw[-{Latex[width=3mm]}] (start) edge node [right]      {Analyse angefordert} (an_start);
\draw[-{Latex[width=3mm]}] (an_start) edge node [above]      {Challenge anfordern} (chal_calc);
\draw[-{Latex[width=3mm]}] (chal_calc) edge [out=260,in=90]node [above] {Challenge senden} (chal_sig);
\draw[-{Latex[width=3mm]}] (chal_sig) edge node [above] {Sig. Chal. senden} (check_chal);
\draw[-{Latex[width=3mm]}] (check_chal) edge node [left] {Analyse starten} (analyse);
\draw[-{Latex[width=3mm]}] (analyse) edge node [above=0.3cm] {Ergebnisse übertragen} (dar_erg);
\draw[-{Latex[width=3mm]}] (dar_erg) edge [out=125,in=235]node [left] {} (start);

\end{tikzpicture}
\caption{Ablauf einer Analyse auf einer separaten VM}
\label{fig:WeitMobSFXMLRPCAblauf}
\end{figure}

Durch das Signieren einer bei jedem Funktionsaufruf neu generierten Challenge ist sichergestellt, dass nur der echte Host die Funktion aufruft und keine \textit{Replay-Attacken} möglich sind.\\

Jedoch gab es bei der Implementierung über \textit{Flask} sowie mit den verschiedenen \textit{Python}-Versionen zwischen Server und Client Probleme beim Datenaustausch. So gibt es in \textit{Python} 2 einen dedizierten Variablen-Typ namens \textit{string}, wohingegen \textit{Python} 3 Strings in codierten Byte-Objekten speichert. Eine Konvertierung ist möglich und wurde umgesetzt. Jedoch legt \textit{Flask} eine weitere Ebene des Encodings über den übertragenen Inhalt, weshalb es leider nicht möglich war, die kryptographische Signatur fehlerfrei zu übertragen. Aus diesem Grund wurde nach weiteren und eventuell besser geeigneten Alternativen gesucht.\\

Nach einer kurzen Suche bot sich das Modul \textit{xmlrpclib} (\textit{Python} 2)/\textit{xmlrpc} (Python 3) an. Es ermöglicht die Kommunikation über das standardisierte XML-RPC-Protokoll\footnote{\url{http://xmlrpc.scripting.com/spec.html}} und ist somit unabhängig von der Python-Version. Zudem können Daten transparent zwischen Server und Client übergeben werden.

Es folgt ein kurzes Beispiel, in welcher ein Client eine \textit{hello\_world}-Funktion am Server aufruft. Dabei ist der Server-Code in Abbildung \ref{lst:WeitMobSFXMLRPCServerCode}, der Client Code in Abbildung \ref{lst:WeitMobSFXMLRPCClientCode} sowie die Ausgabe in Abbildung \ref{lst:WeitMobSFXMLRPCClientAusgabe} dargestellt.\\

\begin{figure}
\begin{lstlisting}
from xmlrpc.server import SimpleXMLRPCServer

def hello_world(name):
	"""Return an Hello-World for a name"""
    return "Hello World {}!".format(name)

if __name__ == '__main__':
	# Open the Server on port 8000
    server = SimpleXMLRPCServer(("0.0.0.0", 8000))
    server.register_function(hello_world, "hello_world")
\end{lstlisting}
\caption{XML-RPC Server Code}
\label{lst:WeitMobSFXMLRPCServerCode}
\end{figure}

\begin{figure}
\begin{lstlisting}
import xmlrpclib

proxy  = xmlrpclib.ServerProxy(
                "http://{}:{}".format(
                    TARGET_IP, 8000
                )
            )
print proxy.hello_world("John Doe")
\end{lstlisting}
\caption{XML-RPC Client Code}
\label{lst:WeitMobSFXMLRPCClientCode}
\end{figure}

\begin{figure}
\begin{lstlisting}
Hello World John Doe!
\end{lstlisting}
\caption{Ausgabe des Clients}
\label{lst:WeitMobSFXMLRPCClientAusgabe}
\end{figure}

Durch den minimalen Eingriff von \textit{xmlrpc} in die Kommunikation konnte die kryptographische Signatur als \textit{base64} codiertes Datum übergeben werden. Die Kern-Funktionen des Clients und Servers sind in den Abbildungen \ref{ref:WeitEntXMLClient} und \ref{ref:WeitEntXMLServer} dargestellt.\\

\begin{figure}
\begin{lstlisting}
def _get_token():
    """Get the authentication token for windows vm xmlrpc client."""
    challenge = proxy.get_challenge()
    priv_key = rsa.PrivateKey.load_pkcs1(
        open(settings.WINDOWS_VM_SECRET).read()
    )
    signature = rsa.sign(challenge, priv_key, 'SHA-512')
    sig_b64 = base64.b64encode(signature)
    return sig_b64
    
print proxy.test_challenge(_get_token())
\end{lstlisting}
\caption{Client-Code}
\label{ref:WeitEntXMLClient}
\end{figure}

\begin{figure}
\begin{lstlisting}
def _check_challenge(signature):
    signature = base64.b64decode(signature)
    try:
        rsa.verify(challenge.encode('utf-8'), signature, pub_key)
        print("[*] Challenge successfully verified.")
        _revoke_challenge()
    except rsa.pkcs1.VerificationError:
        print("[!] Received wrong signature for challenge.")
        raise Exception("Access Denied.")
    except (TypeError, AttributeError):
        print("[!] Challenge already unset.")
        raise Exception("Access Denied.")

def get_challenge():
    """Return an ascii challenge to validate authentication in _check_challenge."""
    global challenge
    # Not using os.urandom for Python 2/3 transfer errors
    challenge = ''.join(
        random.SystemRandom().choice(string.ascii_uppercase + string.digits) for _ in range(256)
    )
    return "{}".format(challenge)
    
def test_challenge(signature):
    """Test function to check if rsa is working."""
    _check_challenge(signature)
    print("Check complete")
    return "OK!"
\end{lstlisting}
\caption{Server-Code}
\label{ref:WeitEntXMLServer}
\end{figure}

Durch diese Art der Implementierung ist eine sichere, zuverlässige, effiziente und leicht erweiterbare Kommunikation zwischen Host und virtueller Maschine möglich.

\newpage
\subsection{Windows-Apps}
\label{Windows-Apps}
Seit Windows 8 und der damit eingeführten \textit{Unified Windows Platform}\footnote{\url{https://docs.microsoft.com/de-de/windows/uwp/get-started/universal-application-platform-guide}} sind moderne Windows-Apps unter Windows sowohl auf einem PC als auch auf einem Handy lauffähig.\\

Bisher hat das \textit{MobSF} noch keine Möglichkeit zur Prüfung von Windows-Apps bereitgestellt. Im Folgenden sind die im Rahmen dieser Arbeit implementierten Features beschrieben.

\subsubsection{Windows Phone Formats}
Um eine App analysieren zu können, muss zu allererst das File-Format betrachtet und verarbeitet werden. Leider sind im Windows-Umfeld diverse Formate gängig, von welchen im Folgenden einige in Hinblick auf Aufbau und Schutz analysiert werden.

\paragraph{XAP}
\textit{XAP} ist ein Format für \textit{Windows-Phone}-Apps ab \textit{Windows Phone 7} und enthält oft \textit{Silverlight}-Applikationen\footnote{\url{https://www.microsoft.com/silverlight/}}. Dementsprechend ist der Mime-Type zumeist \textit{application/x-silverlight-app}.\\

Ursprünglich war ein \textit{XAP}-File einfach ein \textit{ZIP}-Archiv, welches alle Dateien der App umfasst. Aus einem solchen \textit{XAP}-File konnte der Inhalt über folgende Schritte einfach gewonnen werden:
\begin{enumerate}
	\item \textit{XAP}-File herunterladen
	\item evtl. Dateiendung von "`.xap"' auf "`.zip"' ändern
	\item mit einem gängigen Archiv-Programm (z.B. 7-Zip) entpacken
\end{enumerate}
Anschließend liegen allen Dateien der App im Extraktion-Ordner.\\

\paragraph{APPX}
\textit{APPX} ersetzt ab Windows 8.1 das \textit{XAP}-Format. Auch das \textit{APPX}-Format nutzt einen Kompressionsalgorithmus ähnlich zu \textit{Zip}. Daher können Inhalte über das in Python integrierte Modul \textit{zipfile} extrahiert werden. Dies ist in Abbildung \ref{lst:WeitMobSFWinAPPXExtract} dargestellt.\\

\begin{figure}
\begin{lstlisting}
import zipfile
files=[]
with zipfile.ZipFile(APP_PATH, "r") as z:
        z.extractall(EXT_PATH)
        files=z.namelist()
return files
\end{lstlisting}
\caption{Beispiel-Code für das entpacken einer Windows-App}
\label{lst:WeitMobSFWinAPPXExtract}
\end{figure}

Es wurde folgende Mime-Types für APPX-Dateien festgestellt:
\begin{itemize}
    \item \textit{application/octet-stream}
    \item \textit{application/vns.ms-appx}
    \item \textit{application/x-zip-compressed}
\end{itemize}

\paragraph{APPXBundle}
Das Format \textit{appxbundle} ist ein Zusammenschluss mehrerer \textit{APPX}-Dateien. Entsprechend ist der Mime-Type zum Beispiel \textit{application/zip} und kann ebenfalls mit zum Beispiel dem Pyhton-Modul \textit{zipfile} oder einem gängigen Zip-Extraktions-Programm entpackt werden.

\paragraph{DRM}
Leider sind Apps aus dem Windows-Store häufig durch sogenanntes \textit{DRM} (\textit{Digital Rights Management}), hier \textit{Play Ready}, geschützt. Die Apps sind demnach nicht nur durch Zertifikate vor Veränderungen geschützt, sondern zusätzlich sind die App-Dateien mit \textit{AES} im \textit{CTR}-Mod verschlüsselt. Dies ist im \textit{PlayReady}-Header festgelegt, folgend ein Beispiel:
\begin{lstlisting}
<WRMHEADER xmlns="http://schemas.microsoft.com/DRM/2007/03/PlayReadyHeader" version="4.0.0.0">
  <DATA>
    <PROTECTINFO>
      <KEYLEN>16</KEYLEN>
      <ALGID>AESCTR</ALGID>
    </PROTECTINFO>
    <KID>5zhQkM1z5kq6HCCYD9nceQ==</KID>
    <LA_URL>http://microsoft.com/</LA_URL>
    <CUSTOMATTRIBUTES xmlns="">
      <S>rtXfkbbz4yuPNGrzjQc9yA==</S>
      <KGV>0</KGV>
    </CUSTOMATTRIBUTES>
    <CHECKSUM>TpkeZrwUjIY=</CHECKSUM>
  </DATA>
</WRMHEADER>
\end{lstlisting}

Zwar ist es möglich, \textit{DRM} geschützte Apps auf einem gerooteten Windows-Phone zu installieren und anschließend über Apps wie dem \textit{ProgramManager}\footnote{\url{https://forum.xda-developers.com/showthread.php?t=1922454}} als ungeschütztes Archiv an einen PC zu übertragen, jedoch könnte dies als Umgehung eines Kopierschutzes gesehen werden und wurde daher in dieser Arbeit nicht weiter verfolgt.

\newpage
\subsubsection{Virtuelle Maschine zur Analyse von Windows-Apps}
Da sowohl für die dynamische Analyse als auch aufgrund bestimmter Tools eine \textit{Windows}-VM notwendig ist, wird der Aufbau für \textit{MobSF} im Folgenden kurz allgemein dargestellt.\\

Als erstes muss ein Betriebssystem für die Analyse-VM gewählt werden. Der \textit{Windows-Phone}-Simulator steht ab \textit{Windows 8.1 64-Bit} zur Verfügung. Daher sollte \textit{Windows 8.1 64-Bit} oder höher verwendet werden. Hier wurde \textit{Windows 10 64-Bit} verwendet.\\

Für Tests bezüglich der dynamische Analyse wurde \textit{Visual Studio 2015} Community Edition verwendet. Da diese kostenlos ist, sind bis auf das Betriebssystem sind keine kostenpflichtigen Programme beteiligt.\\

Für die Kommunikation zwischen \textit{MobSF} und der VM wurde wurde ein Setup-Skript (genauer beschrieben in Abschnitt \ref{ref:WeitMobSFWindSetup}) angefertigt, das die notwendigen Programme herunterlädt und die Installationen anstößt. Das Skript ist im Anhang unter \ref{ap:simcontrol} zu finden.

\subsubsection{Setup-Skript}\label{ref:WeitMobSFWindSetup}
Die Datei \textit{setup.py} wird über zwei verschiedene Wege aufgerufen, je nachdem ob \textit{MobSF} vollständig auf \textit{Windows} installiert wird oder nur die statische Analyse auf dem \textit{Windows}-System ausgeführt werden soll. Je nach Aufruf werden verschiedene Arbeitsschritte ausgeführt und auch verschiedene \textit{Python}-Versionen verwendet, weshalb manche Funktionen sowohl \textit{Python} 2- wie 3- kompatibel gestaltet sind. Dies führte im Rahmen dieser Arbeit zu einem erhöhten Implementierungsaufwand, jedoch muss kein Weiterentwicklungsaufwand betrieben werden, wenn der Kern von \textit{MobSF} ebenfalls auf \textit{Python} 3 portiert wird.

\paragraph{MobSF auf Windows}
Wird \textit{MobSF} vollständig auf \textit{Windows} installiert, wird das Setup-Script einmalig aus der \textit{settings.py} aufgerufen. Dabei wird \textit{Python} 2 verwendet. Anfangs wird das mit dem Download mitgelieferte Config-File in den richtigen Ordner des User-Kontext kopiert. Anschließend wird das Config-File über das ConfigParser-Modul geladen. Daraufhin werden alle notwendigen Ordner berechnet, in das Config-File geschrieben und angelegt. Damit sind die Vorbereitungen bezüglich der Config und der Ordner abgeschlossen.\\

Anschließend wird \textit{nuget} heruntergeladen. \textit{nuget} ist ein Paket-Manager für Windows, über welchen später \textit{BinSkim} installiert wird. Der Download ist in Python relativ einfach zu implementieren, wie in Listing \ref{lst:WeitNugetDown} dargestellt ist.\\

\begin{figure}
\begin{lstlisting}
# Open File
nuget_file_local = open(
	os.path.join(mobsf_subdir_tools, nuget_file_path), 
	"wb"
)

# Downloading File
print("[*] Downloading nuget..")
nuget_file = urlrequest.urlopen(nuget_url)

# Save content
print("[*] Saving to File {}".format(nuget_file_path))

# Write content to file
nuget_file_local.write(bytes(nuget_file.read()))

# Aaaand close
nuget_file_local.close()
\end{lstlisting}
\caption{Nuget-Download}
\label{lst:WeitNugetDown}
\end{figure}

Nachdem der Download abgeschlossen ist, wird \textit{BinSkim} über \textit{nuget} installiert. Dazu wird \textit{nuget} mit verschiedenen Parametern aufgerufen (siehe Listing \ref{lst:WeitNugetInst}).\\

\begin{figure}
\begin{lstlisting}
# Execute nuget to get binkim
output = subprocess.check_output(
    [
        nuget,
        "install", binskim_nuget, '-Pre',
        '-o', mobsf_subdir_tools
    ]
)
\end{lstlisting}
\caption{Installation von BinSkim über Nuget}
\label{lst:WeitNugetInst}
\end{figure}

Um später eine optimale Ausführung auf sowohl 64- wie 32-Bit Windows-System zu gewährleisten, werden aus den installierten Dateien die Binaries für \textit{BinSkim} \textit{x86} sowie \textit{x64} gesucht und in der Config-Datei gespeichert.\\

Als letztes Tool wird \textit{BinScope} installiert. Dies ist leider nicht über \textit{nuget} möglich. Daher muss eine \textit{MSI}-Datei heruntergeladen und anschließend installiert werden. Dabei muss der Installationspfad entsprechend gesetzt werden. Da dies nicht eindeutig von Microsoft dokumentiert ist, kann der Quellcode dem Appendix unter \ref{ap:BinScopeInstaller} entnommen werden.\\

Um sicherzustellen, dass das Setup-Script nicht mehrmals ausgeführt wird, wird zuletzt ein Lock-File platziert. Bei erneutem Start der Anwendung wird auf dieses geprüft und, falls es existiert, eine erneute Installation übersprungen.

\paragraph{Statische Analyse auf Windows}
Wird nur die statische Analyse auf \textit{Windows} durchgeführt, wird das Setup-Script direkt mit \textit{Python} 3 ausgeführt. Dabei wird ähnlich zur Installation auf Windows zuerst das Config-File initialisiert und anschließend werden die Ordner erstellt. Auch werden \textit{nuget}, \textit{BinSkim} und \textit{BinScope} installiert.\\

Die Ordnerstruktur wurde dabei anfangs wie folgt aufgebaut:

  \dirtree{%
.1 C:/.
    .2 [...].
    .2 MobSF/.
    	.3 Config/.
    		.4 config.txt.
    	.3 Download/.
    	.3 Tools/.    	
}

Um die Kompatibilität mit verschiedenen Systemkonfigurationen zu erhöhen, wurden später statt "`C"' die Ordner des Benutzers für die Speicherung der Programme verwendet.\\

Die \textit{config.txt} enthält Inhalte, welche zentral abgelegt werden und für verschiedene Skripte eine wichtige Rolle spielen. Ein Beispiel wäre der Pfad zum Verzeichnis, in welchem die Tools gespeichert werden. Der Download-Ordner enthält die durch das Setup-Skript heruntergeladenen Binaries, der Tools-Ordner die installierten Tools.\\

Zusätzlich zur lokalen Installation wird ein XMLRPC-Server unter Tools installiert und konfiguriert. Die genaue Funktionsweise dieses Services ist unter \ref{ref:XMLRPC} beschrieben.\\

Von Ajin Abraham, einem Contributor zu \textit{MobSF}, wurde nach der Implementierung ein ergänzendes Video zur Installation erstellt und auf Youtube veröffentlicht\footnote{\url{https://www.youtube.com/watch?v=17ilENuMj58}}.

\newpage
\subsubsection{Statische Analyse}
Um die drei marktführenden mobilen Betriebssysteme mit \textit{MobSF} abzudecken, wurden Funktionen für Windows-Phone-Apps hinzugefügt.\\

Zur statischen Analyse wurde das Tool \textit{binskim} von Microsoft getestet \footnote{\url{https://github.com/Microsoft/binskim/releases}}. Das Tool analysiert Compiler-Flags und verschiedenste andere statisch feststellbare Eigenschaften, bewertet diese und gibt die Ergebnisse im SARIF-Format\footnote{\url{https://github.com/sarif-standard/sarif-spec/}} zurück. Leider ist das Tool nur unter Windows ausführbar. Da \textit{MobSF} jedoch auch auf Linux und Mac OS X lauffähig sein soll, wird im Folgenden eine virtuelle Windows-Maschine zur statischen und dynamischen Analyse verwendet.

%\_CRT\_SECURE\_NO\_WARNINGS

\paragraph{Files}
Ein einfacher, aber oft sehr hilfreicher erster Eindruck entsteht durch die Auflistung der enthaltenen Files. In \textit{MobSF} existiert daher eine Methode, welche \textit{ZIP}-Archive entpackt und Pfad sowie Name der entpackten Dateien zurückliefert. Über diese Methode wurden auch hier die enthaltenen Files erfasst und in der Oberfläche dargestellt.

\paragraph{Bad Functions}\label{ref:WeitEntwWinBadFunc}
Ebenso wie bei \textit{iOS} wurde auch hier eine Extraktion der Strings aus dem Binary implementiert. Genutzt wurde dafür die unter \ref{ref:strings} entwickelte Implementierung des Strings-Kommandos in Python.\\

In diesen Strings sind ebenfalls die Namen der verwendeten Funktionen vorhanden, sodass jetzt auch hier eine Suche nach bekannterweise verwundbaren Funktionen implementiert wurde.

\newpage
\subsubsection{Zusätzlich eingebundene Tools zur statischen Analyse}\label{ref:WeitMobEingTools}
Im Folgenden werden die für die Analyse von Windows-Apps eingebundenen Tools vorgestellt.

\paragraph{BinSkim}
BinSkim ist ein Tool des \textit{Microsoft SDL} (\textit{Secure Development Lifecycle}) und prüft \textit{Windows}-Applikationen bezüglich der Konfiguration des Compilers/Linkers sowie andere Security-Relevante Merkmale.\cite{BinSkimGithub}\\

Dabei können folgende Fehlkonfigurationen erfasst werden:
\begin{description}
	\item[LoadImageAboveFourGigabyteAddress] 64-bit Images sollten eine \textit{base address} über der 4GB Grenze nutzen, um den Kompatibilitätsmodus von \textit{ASLR} nicht zu aktivieren. Dieser würde zu einem kleineren Raum für \textit{ASLR} und somit zu niedrigerer Sicherheit führen.
	
	\item[DoNotIncorporateVulnerableDependencies] Binaries sollten keine Abhängigkeiten zu Code mit bekannten Schwachstellen haben.
	
	\item[DoNotShipVulnerableBinaries] Es sollten keine Libraries enthalten sein, welche bekannte Schwachstellen besitzen.
	
	\item[BuildWithSecureTools] Applikationen sollten mit den aktuellsten Compiler- und Tool-Versionen kompiliert werden, um alle Sicherheitsfunktionen nutzen zu können.
	
	\item[EnableCriticalCompilerWarnings] Binaries sollten mit einem Warn-Level kompiliert werden, welcher alle Sicherheit-relevanten Checks angibt.
	
	\item[EnableControlFlowGuard] Binaries sollten das \textit{compiler control guard feature} (\textit{CFG}) zur BuildZeit aktivieren, um Angreifer davon abzuhalten, den Programmablauf zu verändern.
	
	\item[EnableAddressSpaceLayoutRandomization] Binaries sollten als \textit{DYNAMICBASE} gelinkt sein, um an \textit{ASLR} teilnehmen zu können.
	
	\item[DoNotMarkImportsSectionAsExecutable] PE-Import-Sektionen sollten nie gleichzeitig als schreibbar und ausführbar markiert sein.
	
	\item[EnableStackProtection] Binaries sollten \textit{stack protection} (zum Beispiel \textit{Stack-Canaries} über die \textit{/GS}-Flag) nutzen, um das Ausnutzen von \textit{stack buffer overflows} zu erschweren.
	
	\item[DoNotModifyStackProtectionCookie] Applikations-Code sollte keinen Einfluss auf die Stack-Absicherung haben.
	
	\item[InitializeStackProtection] Binaries sollten die Stack-Absicherung korrekt initialisieren, um das Ausnutzen von \textit{stack buffer overflows} zu erschweren.
	
	\item[DoNotDisableStackProtectionForFunctions] Applikations-Code sollte keine Stack-Absicherung für einzelne Funktionen abschalten.
	
	\item[EnableHighEntropyVirtualAddresses] Binaries sollten als kompatibel mit \textit{high entropy ASLR} markiert werden.
	
	\item[MarkImageAsNXCompatible] Binaries sollten das \textit{NX}-Bit gesetzt haben, um ungewollte Ausführung von Daten als Code zu verhindern.
	
	\item[EnableSafeSEH] X86 Binaries sollten \textit{SafeSEH} aktivieren, um das Ausnutzen von Memory-Schwachstellen zu erschweren.
	
	\item[DoNotMarkWritableSectionsAsShared] Code- und Data-Sections sollten nicht sowohl als \textit{shared} und \textit{writeable} markiert sein.
	
	\item[DoNotMarkWritableSectionsAsExecutable] PE-Sections sollten nicht sowohl \textit{writeable} wie auch \textit{executeable} sein.
	
	\item[SignSecurely] Images sollten über sichere kryptographische Signaturen vom einen vertrauenswürdigen Author geschützt sein.
\end{description}
$ $\\
Der originale Text sowie längere Beschreibungen sind dem Output des Kommandos
\begin{lstlisting}
BinSkim.exe exportRules output.json
\end{lstlisting}
zu entnehmen.

\paragraph{BinScope}
\textit{BinScope} ist ebenfalls ein Programm des \textit{Microsoft Secure Development Lifecycle} und untersucht ebenfalls Binaries auf fehlerhafte Konfigurationen. Es ist der Vorgänger zu \textit{BinSkim} und hat daher ähnliche Fehlerkategorien, welche im Folgenden dargestellt sind. Leider sind die Checks nicht dokumentiert, sodass der Autor aus den Namen und den bisher beobachteten Ergebnissen auf deren Funktionsweise schließt. \\

Im Gegensatz zu \textit{BinSkim} braucht \textit{BinScope} weniger Konfigurationsaufwand und ist daher, obwohl es der Vorgänger von \textit{BinSkim} ist, durchaus eine sinnvolle Ergänzung.

%\paragraph{Decompiler?}
%http://www.telerik.com/products/decompiler.aspx

\begin{description}
	\item[ATLVersionCheck] stellt sicher, dass die ATL-Header, welche für den Build des Binaries verwendet wurden, in Ordnung sind. Diese Regel kommt nur bei Dateien vom COM-Type zum Einsatz.
	
	\item[ATLVulnCheck] prüft, ob Klassen, welche IPersistStreamInit implementieren, potenziell verwundbare Eigenschaften aufweisen. Diese Regel kommt nur bei Dateien vom COM-Type zum Einsatz.
	
	\item[AppContainerCheck] prüft, ob die App mit der AppContainer-Flag kompiliert wurde. Die Flag ermöglicht eine eine isolierte Umgebung zur Ausführung der App\footnote{\url{https://msdn.microsoft.com/en-us/library/windows/desktop/mt595898(v=vs.85).aspx}}.
	
	\item[CompilerVersionCheck] prüft, ob für die App eine aktuelle Kompilerversion genutzt wurde (mindestens 14.00.50727).
	
	\item[DBCheck] prüft, ob das Binary mit der \textit{Dynamic Base}-Flag kompiliert wurde, also an \textit{ASLR} teilnehmen kann.
	
	\item[GSCheck] verifiziert, dass die \textit{/GS-Flag} beim Kompiliervorgang verwendet wurde, welche den \textit{Stack} vor \textit{Memory-Corruption} schützt (z.B. über \textit{Stack-Canaries}). 
	
	\item[DefaultGSCookieCheck] ist ein Check zur Konfiguration von GS, welcher die Art der verwendeten Cookies überprüft.
	
	\item[GSFriendlyInitCheck] ist ein Check zur Konfiguration von GS, welcher prüft, ob das Binary entsprechende GS-freundliche Einstiegspunkte bietet.
	
	\item[GSFunctionSafeBuffersCheck] ist ein weiterer Check zur Konfiguration von GS, welcher prüft, ob auch die Funktions-Buffer durch GS abgesichert werden.
	
	\item[ExecutableImportsCheck] prüft, ob die Import-Section als ausführbar markiert ist.
	
	\item[FunctionPointersCheck] prüft auf \textit{global function pointers}. Durch das Überschreiben von statischen Buffern können evenutel auch \textit{global function pointer} überschrieben werden, was eine Schwachstelle darstellen kann.

	\item[HighEntropyVACheck] prüft, ob für ASLR eine hohe Entropie zur Verfügung steht.
	
	\item[NXCheck] prüft, ob das NX-Flag (stehend für "`no execute"') gesetzt ist. Durch diese Maßnahme werden Exploits erschwert, da bestimme Speicherbereiche als nicht ausführbar gekennzeichnet werden.
	
	\item[RSA32Check] prüft auf 32-Bit RSA-Keys, für welche eine Primfaktorzerlegung leicht möglich wäre.
	
	\item[SafeSEHCheck] prüft, ob SafeSEH aktiviert wurde. Die Verwendung von SafeSEH verbessert das Fehlerhandling und erschwert die Entwicklung von Exploits.
	
	\item[SharedSectionCheck] prüft, ob Sections sowohl als \textit{shared} als auch \textit{writeable} markiert sind.
	
	\item[VB6Check] prüft, ob VB6-Code zum Einsatz kommt.
	
	\item[WXCheck] prüft, ob die WX-Flag verwendet wurde. Durch diese Flag müssen Linker-Warnungen wie Fehler behandelt werden, wodurch die Sicherheit erhöht wird.
\end{description}

\newpage
\subsection{iOS-Apps}
Ein weiteres wichtiges Betriebssystem ist das auf \textit{iPhones} verwendete \textit{iOS}.  Es hat einen Marktanteil von ca. $20\%$\cite{KatarWorldpanelMobBetrSys} und wird auch von Unternehmen immer häufiger genutzt. Umso wichtiger ist es, dass angebotene Apps entsprechend auf deren Sicherheit geprüft werden können. Daher sind im Folgenden durchgeführte Erweiterungen zu \textit{MobSF} aufgezeigt.

\subsubsection{IPA- und APP-Format}
Um eine \textit{iOS}-App zu analysieren, muss diese in einem passenden Format vorliegen. Geläufig und bereits von \textit{MobSF} unterstützt ist das \textit{IPA}-Format. Nativ werden \textit{iOS}-Apps jedoch als \textit{.app}-Dateien abgelegt. Die Umwandlung von \textit{.app} zu \textit{.ipa} ist über wenige händische Schritte zu verwirklichen. Kompiliert man eine App in \textit{Xcode}, wird diese unter einem bestimmten Pfad abgelegt. Der Standardpfad unter \textit{Xcode} 8 ist \textit{/Users/<username>/Library/Developer/Xcode/DerivedData/}. Alternativ kann der Pfad den Projekt-Einstellung in \textit{Xcode} über \textit{File $\rightarrow$ Projects Settings} entnommen werden. Die Struktur des \textit{DerivedData}-Ordner ist unter \ref{ref:deriveddata} dargestellt.\\

\begin{figure}
\dirtree{%
.1 DerivedData/.
    .2 \textit{Appname}-\textit{random}/.
    .2 \textit{Appname2}-\textit{random}/.
    	.3 Build/.
    		.4 Products/.
    			.5 \textit{Platform}/.
    				.6 \textit{Appname}.app.
    	.3 Index/.
    	.3 Logs/.
    	.3 TextIndex/.
    	.3 info.plist.
    	.3 scm.plist.    	
}
\caption{Struktur des DerivedData-Ordners}
\label{ref:deriveddata}
\end{figure}

Die \textit{.app}-Datei kann nun per Drag-n-Drop in den App-Bildschirm von \textit{iTunes} gezogen werden. Anschließend wird die App in \textit{iTunes} angelegt (siehe Grafik \ref{fig:itunes_app}). Wird die App per Drag-n-Drop wieder aus \textit{iTunes} in den Finder gezogen, wird die App als IPA-File abgelegt.

\begin{figure}[htbp]
	\centering
	\includegraphics[width=\textwidth]{bilder/pentest_mobile_anwendungen/weiterentw_mobsf/itunes_app.png}
	\caption{iTunes-App-Fenster}
	\label{fig:itunes_app}
\end{figure}

\subsubsection{iOS-Permissions}
Eine wichtige Information über \textit{iOS}-Apps ist, welche Berechtigungen diese erfordern. In der aktuelle Version von \textit{iOS} müssen Berechtigungen, welche die App zur Laufzeit anfordert, in der \textit{info.plist} angekündigt werden. Im Folgenden sind die derzeit möglichen Berechtigungen aufgeführt.

\newpage
\begin{description}
	\item[NSAppleMusicUsageDescription]  beschreibt, dass die App den Zugriff auf die Musik-Bibliothek anfordern kann.
	
	\item[NSBluetoothPeripheralUsageDescription] beschreibt, dass die App die Berechtigung anfordern kann, auf das Bluetooth-Interface zuzugreifen.
	
	\item[NSCalendarsUsageDescription] beschreibt, dass die App den Zugriff auf den Kalender anfordern kann. Dadurch können alle Termine gelesen und verändert sowie gelöscht werden.
	
	\item[NSCameraUsageDescription] beschreibt, dass die App den Zugriff auf die Kamera anfordern kann. Dadurch können Fotos und Videos aufgenommen werden.
	
	\item[NSContactsUsageDescription] beschreibt, dass die App den Zugriff auf die Kontakte des User anfordern kann. Dadurch können alle Einträge im Kontaktbuch gelesen, verändert und gelöscht werden. 
	
	\item[NSHealthShareUsageDescription] beschreibt, dass die App den Lese-Zugriff auf die durch das Gerät erfassten Gesundheitsdaten anfordern kann. Diese können zum Beispiel zurückgelegte Strecken oder die Herzfrequenz enthalten.
	
	\item[NSHealthUpdateUsageDescription] beschreibt, dass die App den Schreib-Zugriff auf die Gesundheitsdaten des Users anfordern darf.
	
	\item[NSHomeKitUsageDescription] beschreibt, dass die App den Zugriff auf das Home-Kit des Users anfragen kann. Weitere Details zum Home-Kit sind auf der Apple-Webseite\footnote{\url{http://www.apple.com/de/ios/home/}} zu finden.
	
	\item[NSLocationAlwaysUsageDescription] beschreibt, dass die App das Recht anfragen kann, zu jeder Zeit den GPS-Standort des Users erfassen.
	
	\item[NSLocationWhenInUseUsageDescription] beschreibt, dass die App das Recht anfragen kann, den GPS-Standort auszulesen, solange die App im Vordergrund aktiv ist.
	
	\item[NSMicrophoneUsageDescription] beschreibt, dass die App das Recht anfragen kann, auf das Mikrophon zuzugreifen. 
	
	\item[NSMotionUsageDescription] beschreibt, dass die App den Zugriff auf den Motion-Sensor anfragen kann. Dadurch können Bewegungen erfasst werden.
	
	\item[NSPhotoLibraryUsageDescription] beschreibt, dass die App den Zugriff auf die Photo-Bibliothek anfordern kann.
	
	\item[NSRemindersUsageDescription] beschreibt, dass die App den Zugriff auf die Reminder des Users anfordern kann. Dadurch können Reminder ausgelesen, erstellt, verändert oder gelöscht werden.
	
	\item[NSVideoSubscriberAccountUsageDescription] beschreibt, dass die App den Zugriff auf den TV-Account des Users anfragen kann.
\end{description}

Diese, sowie weitere Einträge für die \textit{Info.plist} sind der Apple-Dokumentation zu entnehmen \footnote{\url{https://developer.apple.com/library/content/documentation/General/Reference/InfoPlistKeyReference/Articles/CocoaKeys.html}}.\\

In der Auflistung wurde die Formulierung "`anfragen kann"' genutzt, da der Eintrag in der \textit{info.plist} den App nicht automatisch die Rechte gibt, sondern der User diese bestätigen muss, sobald die App diese wirklich anfordert. Jedoch können zur Laufzeit keine Rechte angefordert werden, welche nicht vorher in der \textit{info.plist} festgelegt wurden.\\

Um in \textit{MobSF} die Berechtigungen auszulesen, wurde dementsprechend die \textit{info.plist} ausgelesen und auf die entsprechenden Einträge geprüft. Da die Datei jedoch im Binärformat vorliegt, muss diese zuerst umgewandelt werden. Dies kann über das in \textit{Xcode} enthaltene Tool \textit{plutil} erreicht werden. Der Aufruf ist dabei wie folgt:
\begin{lstlisting}
plutil -convert xml1 info.plist
\end{lstlisting}
Anschließend kann über das in Python enthaltene Modul \textit{plistlib} wie folgt auf die Einträge zugegriffen werden:
\begin{lstlisting}
p_list = plistlib.readPlistFromString(read_bin_xml(converted_info_plist_file))
if "NSBluetoothPeripheralUsageDescription" in p_list:
        print("Bluetooth-Permission found!")
\end{lstlisting}

\pagebreak
In \textit{MobSF} wird auf diese Art die \textit{info.plist}-Datei verarbeitet und die Ergebnisse in der Web-Oberfläche dargestellt. Das Ergebnis für eine App mit Bluetooth-Berechtigung ist in Grafik \ref{fig:permission_check} dargestellt.\\

Auf einige Rechte wie den Zugriff auf Kontakte, Kalender, Reminder, Kamera und Mikrophon sollte bei Analyse besonders geachtet werden.

\subsubsection{Erkennung von ungesicherten Verbindungen}\label{ref:WeitMobSFErkennungVonUngesichertenVerbindungen}
Wie in \ref{ref:inseccon} unter "`Ungesicherte Verbindungen"' beschrieben, müssen ab \textit{iOS} 9.0 Apps den  RFC-Standard 2818\footnote{\url{https://tools.ietf.org/html/rfc2818}} nutzen, um Verbindungen zu Webseiten oder APIs aufzubauen.\\

Daher wurde \textit{MobSF} um ein Feature ergänzt, welches die \textit{Info.plist} auf Ausnahmen überprüft. Der Code ist unter Abbildung \ref{lis:NSAppTransportSecurity} dargestellt.\\

\begin{figure}
	\begin{lstlisting}
def __check_insecure_connections(p_list):
    '''Check info.plist for insecure connection configurations.'''
    print "[INFO] Checking for Insecure Connections"

    insecure_connections = []

    if 'NSAppTransportSecurity' in p_list:
        ns_app_trans_dic = p_list['NSAppTransportSecurity']
        if 'NSExceptionDomains' in ns_app_trans_dic:
            for key in ns_app_trans_dic['NSExceptionDomains']:
                insecure_connections.append(key)

    return insecure_connections
	\end{lstlisting}
	\caption{Auslesen von Ausnahmen bezüglich der TLS-Konfiguration aus der Info.plist}
	\label{lis:NSAppTransportSecurity}
\end{figure}

\begin{figure}[htbp]
	\centering
	\includegraphics[width=\textwidth]{bilder/pentest_mobile_anwendungen/weiterentw_mobsf/perm_con_check.png}
	\caption{Ergebnis für eine App mit Bluetooth-Berechtigung und einer Ausnahme der \textit{ATS}}
	\label{fig:permission_check}
\end{figure}


Werden Ausnahmen gefunden, werden diese in der HTML-Oberfläche dargestellt. Ein Beispiel ist in \ref{fig:permission_check} zu sehen.


\subsubsection{Dynamische Analyse über Simulator}
Im Rahmen dieser Masterarbeit wurden ebenfalls Möglichkeiten getestet, wie eine dynamische Analyse von \textit{iOS}-Apps über den "`Simulator"' abgebildet werden könnte. Um dieses Ziel zu erreichen, muss zuerst eine Automatisierung des Simulators möglich sein. Diese Automatisierung kann über das in \textit{Xcode} enthaltene Tool "`xcrun"' realisiert werden.\\

%Einige für die Automatisierung relevante Parameter sind im Folgenden aufgeführt.
%\begin{description}
%	\item[boot] startet ein bereits existierendes virtuelles Gerät.
%	\item[shutdown] fährt ein gestartetes herunter.
%	\item[delete] löscht ein Gerät.
%	\item[openurl] öffnet eine URL auf einem gestarteten Gerät.
%	\item[install] installiert eine App anhand einer \textit{.app} auf einem Gerät.
%	\item[uninstall] deinstalliert eine App vom Gerät.
%	\item[launch] öffnet eine bereits installierte App auf einem Gerät.
%	\item[terminate] beendet eine bereits gestartete App auf einem Gerät.
%\end{description}

\pagebreak
Über die Befehle von \textit{xcrun} kann folgender Ablauf für die dynamische Analyse realisiert werden.
\begin{enumerate}
	\item Gerät starten (Parameter \textit{boot})
	\item App installieren (Parameter \textit{install})
	\item App ausführen (Parameter \textit{launch})
	\item Analyse durchführen
	\item App schließen (Parameter \textit{termiate})
	\item App deinstallieren (Parameter \textit{uninstall})
	\item Gerät herunterfahren/ggf. löschen (Parameter \textit{shutdown} oder \textit{delete})
\end{enumerate}
Somit sollte für erste Tests ein geeigneter Ablauf zur Verfügung stehen. Alternativ hätte ein Tool von \textit{Facebook}\footnote{\url{https://github.com/facebook/FBSimulatorControl}} genutzt werden können, welches wohl einen ähnlichen Funktionsumfang besitzt. Da das Ziel jedoch eine Eingliederung in MobSF war, wurde die Eigenimplementierung über \textit{xcrun} gewählt.\\

Ein Kerninhalt der dynamischen Analyse ist zumeist der Netzwerkverkehr. Eine simple Aufzeichnung über \textit{TCPDump} oder \textit{Wireshark} kann natürlich leicht vollzogen werden. Um dies jedoch zu verifizieren, wurde eine App mit dem in Abbildung \ref{lis:WeitMobSFiOSOpenURLSec} gezeigten Code angelegt.\\

\begin{figure}[htbp]
\begin{lstlisting}
NSURL *url = [NSURL URLWithString:@"https://api.ipify.org"];
SData *data = [NSData dataWithContentsOfURL:url];
NSString *ret = [[NSString alloc] initWithData:data encoding:NSUTF8StringEncoding];
\end{lstlisting}
\caption{Aufruf von \url{https://api.apify.org}}
\label{lis:WeitMobSFiOSOpenURLSec}
\end{figure}

Die \textit{Wireshark}-Ausgabe verifiziert, dass die Verbindung aufgezeichnet werden kann, wie in Grafik \ref{fig:WeitMobSFiOSWireshark} zu sehen ist.\\

\begin{figure}[htbp]
	\centering
	\includegraphics[width=\textwidth]{bilder/pentest_mobile_anwendungen/weiterentw_mobsf/wireshark_simulator.png}
	\caption{Verbindung zu \url{https://api.ipify.org} aus dem iOS-Simulator}
	\label{fig:WeitMobSFiOSWireshark}
\end{figure}

Aufgrund der strengen Vorkonfiguration bezüglich der Transport-Sicherung (siehe Abschnitt \ref{ref:inseccon}), ist mit mehr TLS-Verkehr als bei gewöhnlichen Anwendungen zu rechnen. Umso wichtiger ist es, diese unterbrechen und den Inhalt entschlüsseln zu können. Ein passendes Tool hierfür ist \textit{MITMProxy}, welches als lokaler Proxy-Server alle TLS-Verbindungen unterbricht. Zum Client hin wird ein selbst generiertes Zertifikat angeboten, während zum Webserver eine valide TLS-Verbindung aufgebaut wird. Dies hat natürlich den Nachteil, dass am emulierten Gerät dem selbst generierten Zertifikat vertraut werden muss. Auch zu prüfen bleiben die Möglichkeiten zur Einstellung des Proxy-Servers auf dem emulierten Gerät.\\

Zuerst wurde die Möglichkeit geprüft, das Zertifikat auf emulierten Geräten in den Trust-Store aufzunehmen. Da eine händische Aufnahme weder möglich, noch für die Automatisierung vorteilhaft wäre, sollte ein passendes Tool genutzt werden. \textit{ADVTOOLS}\footnote{\url{https://github.com/ADVTOOLS/ADVTrustStore}} ermöglicht es, eine beliebige \textit{CA} in den Trust-Store der emulierten Geräte aufzunehmen, und bietet damit die passende Funktionalität.\\

Als nächstes ist die Konfiguration von Proxy-Servern auf den emulierten Geräte zu realisieren. Auch dies ist leider nicht direkt über den Simulator realisierbar, jedoch nutzt der Simulator den Proxy-Server des\textit{ Mac OS X}-Host-Systems. Dieser kann über die System-Einstellungen konfiguriert werden (siehe Grafik \ref{fig:WeitMobSFiOSProxySettings}).\\

\begin{figure}[htbp]
	\centering
	\includegraphics[width=\textwidth]{bilder/pentest_mobile_anwendungen/weiterentw_mobsf/proxy_settings.png}
	\caption{Einstellung des Proxy-Servers unter \textit{Mac OS X}}
	\label{fig:WeitMobSFiOSProxySettings}
\end{figure}

Somit ist nun eine TLS-Unterbrechung über \textit{MITMProxy} denkbar. \textit{MITMProxy} bietet eine \textit{NCurses}-Oberfläche an, um den durchlaufenden Netzwerkverkehr zu analysieren. Diese ist in Grafik \ref{fig:WeitMobSFiOSMITMProxUI} dargestellt.\\

\begin{figure}[htbp]
	\centering
	\includegraphics[width=\textwidth]{bilder/pentest_mobile_anwendungen/weiterentw_mobsf/mitmproxy_ncurses.png}
	\caption{Oberfläche von \textit{MITMProxy}}
	\label{fig:WeitMobSFiOSMITMProxUI}
\end{figure}

Diese Oberfläche ist jedoch für eine automatisierte Auswertung ungeeignet. Als zusätzliches Feature bietet \textit{MITMProxy} seine Funktionalität auch als Python-Modul an. Über diese wurde ein Server implementiert, welcher den Netzwerkverkehr aufzeichnet und in einem geeigneten Format ablegt. Der Server ist im Anhang unter \ref{ap:mitmserver} dargestellt.\\

Insoweit ist eine dynamische Analyse möglich. Der entsprechende Code, um alle Schritte zusammen zu führen, ist ebenfalls im Anhang unter \ref{ap:simcontrol} zu finden. Da jedoch, wie in \ref{ref:VergAktSitiOSArch} beschrieben, nur entsprechend kompilierte Apps oder solche, für welche der Sourcecode zur Verfügung steht, getestet werden können, ist der Use-Case für \textit{MobSF} relativ gering. Daher wird der Sourcecode zwar veröffentlicht, jedoch nicht in \textit{MobSF} integriert.\\

Im Laufe des Jahres soll eine Implementierung der dynamischen Analyse auf Basis eines physikalischen Geräts mit Jailbreak erfolgen. Dies ist jedoch nicht mehr Teil dieser Arbeit.

%http://eightbit.io/post/64319534191/how-to-set-up-an-ios-pen-testing-environment

%remote anzeige
%git://git.saurik.com/cydia.git
%http://kanaka.github.io/noVNC/ als front interface
%http://sharedinstance.net/2013/10/running-tweaks-in-simulator/ vnc server auf Iphone installieren, z.B. 
%TODO: Installation VNC-Server auf Simulator testen <-- Kein Chance
%TODO: VNC in Web-Oberfläche testen
		

  
\section{Abgleich mit Anforderungen}
Durch die Weiterentwicklung des \textit{MobSF} wurde vor allem der Punkt "`Unterstützung von \textit{Android}/\textit{iOS}/\textit{Windows Phone}"' verbessert. So ist nun eine statische Analyse von \textit{Windows-Phone}-Apps über die Tools des \textit{Windows SDL} möglich. Zudem wurde die Grundlage für eine dynamische Analyse von \textit{iOS}-Apps geschaffen. Auch wurde die statische Analyse von \textit{iOS}-Apps erweitert, was bessere Ergebnisse zur Folge hat.\\

Somit sind alle Anforderungen aus \ref{ref:PenMobAnwdWeiterAnford} ausreichend erfüllt.


\section{Anwendung der Umgebung}
	Im Folgenden wird die weiterentwickelte Version des \textit{MobSF} auf zwei selbstgeschriebene Applikationen angewendet. Dabei handelt es sich bei Anwendung 1 um eine \textit{Windows-Phone}-App und bei Applikation 2 um eine \textit{iOS}-App.

	\subsection{Test Anwendung 1 - Windows-Phone}
	Um die eingebauten Features bezüglich der statischen Analyse von Windows-Phone-Apps zu testen, wurde eine minimalistische App geschrieben.\\
	
	Diese enthält ein Code-Segment mit der \textit{strcpy}-Funktion, welches unter \ref{lst:PenMobAnwWinStrcpy} dargestellt ist. Ebenso wurde die Option \textit{CFG} (\textit{Control Flow Guard}) für den Test deaktiviert. Der vollständige Quellcode ist dem Datenträger unter 
\begin{lstlisting}
/Apps/iOS/vuln_app_wp.zip
\end{lstlisting} angehängt. Die Inhalte des Datenträgers sind ebenfalls dem zu dieser Arbeit zugehörigen Repository in Github veröffentlicht\footnote{\url{https://github.com/DominikSchlecht/Pen-Test_bei_mob_Anwendungen}}.\\
	
\begin{figure}
\begin{lstlisting}
char to[10];
char from[] = "AAAAAAAAAAA";
strcpy(to, from);
\end{lstlisting}
\caption{Buffer-Overflow über strcpy in der Windows-Phone-Test-App}
\label{lst:PenMobAnwWinStrcpy}
\end{figure}
	
	Die App wurde anschließend durch \textit{MobSF} analysiert. Dabei wurden mehrere unsichere Funktionen, unter anderem die absichtlich eingebaute \textit{strcpy}-Funktion gefunden. Ebenso wurde detektiert, dass \textit{CFG} nicht aktiviert ist. Der vollständige Bericht, inklusive der String-Analyse, ist auf dem Datenträger unter \begin{lstlisting}
/Apps/iOS/vuln_app_wp_report.pdf
\end{lstlisting} abgelegt.
	
	\subsection{Test Anwendung 2 - iOS}
	Zur Entwicklung der Test-Applikation wurde \textit{Xcode 8} verwendet. Die App hat eine definierte Ausnahme in der \textit{ATS} und benötigt die \textit{NSAppleMusicUsageDescription}-Berechtigung. Ebenso benutzt sie, ähnlich zur \textit{Windows-Phone}-App, \textit{strcpy} zum Kopieren eines \textit{char}-Arrays. Die vollständigen Source-Dateien sind auf dem beigelegten Datenträger unter 
\begin{lstlisting}
/Apps/iOS/vuln_app_ios.zip
\end{lstlisting}
zu finden.\\
	
	Auch hier findet \textit{MobSF} die eingebauten Anomalien. So werden die Ausnahme der \textit{ATS}, sowie die \textit{strcpy}-Funktion gefunden und als unsicher dargestellt. Auch die \textit{NSAppleMusicUsageDescription}-Berechtigung wurde detektiert und mit der Begründung dargestellt. Unter Grafik \ref{fig:PenMobAnwiOSErgebnis} sind die Ergebnisse im Überblick dargestellt; der vollständige Bericht ist auf dem Datenträger unter 
\begin{lstlisting}
/Apps/iOS/vuln_app_ios_report.pdf
\end{lstlisting} abgelegt.
	
\begin{figure}[htbp]
	\centering
	\includegraphics[width=\textwidth]{bilder/pentest_mobile_anwendungen/ergebnis_ios.png}
	\caption{Ergebnis des MobSF für die iOS-App}
	\label{fig:PenMobAnwiOSErgebnis}
\end{figure}