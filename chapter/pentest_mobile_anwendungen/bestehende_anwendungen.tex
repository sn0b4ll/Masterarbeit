\section{Bestehende Anwendungen}
Trotz der relativ neuen Thematik der Mobilen Applikationen gibt es schon einige Programme und Applikationen, die bei der Identifizierung von Schwachstellen helfen können. Im Folgenden sind diese unterteilt in \textit{All-In-One-Framework} und Einzelanwendungen. Die Namen sind hierbei sprechend: Sogenannte \textit{All-In-One-Frameworks} bündeln mehrere kleine Anwendungen und automatisieren den Ablauf oder vereinfachen die Bedienung.

\subsection{All-In-One-Framework: MobSF}
\textit{MobSF} ist das einzige, derzeit öffentlich Verbreitete All-In-One-Framework zur Analyse von Mobilen Applikationen. Es ist eine Plattform zur statischen Analyse von Android und iOS-Apps sowie zur dynamischen Analyse von Android Apps. Es bündelt viele kleinere Anwendungen, welche unter \ref{Pen:Eingelanwendungen} aufgeführt sind, in einer einfachen Weboberfläche. Es ist Open-Source, in \textit{Python} geschrieben und steht in \textit{GIT} frei zur Verfügung.\footnote{\url{https://github.com/ajinabraham/Mobile-Security-Framework-MobSF}} Die aktuelle Version ist \textit{0.9.2 beta}.

Es unterstützt die statische Analyse von Apps in den Formaten \textit{APK} und \textit{IPA} sowie aus einfach komprimierten Archiven (\textit{zip}). Zusätzlich beinhaltet \textit{MobSF} einen eingebauten API Fuzzer und ist in der Lage, API-spezifische Schwachstellen wie XXE, SSRF oder Path Traversal zu erkennen (TODO Auflisten).

\subsection{Einzelanwendungen}\label{Pen:Eingelanwendungen}
Das All-In-One-Framework MobSF greift im Hintergrund oft auf eigenständige Tools zurück. Da es für Penetration-Test oft hilfreich ist, diese ohne ein umgebendes Framework nutzen zu können, sind im Folgenden die wichtigsten Tools kurz aufgeführt.
\\\\
Für Android-Apps:
\begin{description}
	\item[jd-core] ist eine Java Decompiler für Java 5 und spätere Versionen. Er steht unter \url{http://jd.benow.ca/} zur Verfügung und kann zum Beispiel über das auf der selben Seite zur Verfügung gestellte JD-GUI genutzt werden.
	\item[Dex2Jar (d2j)] ist ein Tool zum Umwandeln von \textit{.dex}-Dateien (Dalvik-Bytecode) zu normalen Java-Bytecode (gepackt in einem Jar-File). Anschließend können normale Java-Tools zur Analyse verwendet werden. Das Tool ist kostenlos, Open-Source und in Github\footnote{\url{https://github.com/pxb1988/dex2jar}} verfügbar.
	\item[enjarify ]ist eine modernere alternative zu \textit{Dex2Jar}.  \textit{enjarify} wurde von Google entwickelt, ist jedoch trotzdem unter der Apache-Lizenz in Github veröffentlicht\footnote{\url{https://github.com/google/enjarify}}.
	\item[Dex2Smali] stammt unterstützt ebenfalls bei der Konvertierung von \textit{.dex}-Files in andere Formate. In diesem Fall ist das Zielformat \textit{smali}. Dieses Tool ist ebenfalls Open-Source und in Github\footnote{\url{https://github.com/JesusFreke/smali}} zu finden.
	
	\item[procyon ] ist ein Framework zur Analyse von Java-Bytecode. Insbesondere ist ein Decompiler enthalten, welcher den Bytecode in lesbaren Java-Code umwandelt. Das Tool ist kostenlos auf Bitbucket\footnote{\url{https://bitbucket.org/mstrobel/procyon/overview}} verfügbar.
\end{description}
$ $\\\\
Für iOS-Apps:
\begin{description}
	\item[otool ], auch "`object file displaying tool"' genannt, ist ein Tool zur Analyse von Object-Files. Es ist auf Mac OS X bei der Installation von XCode enthalten. Es bietet viele brauche Funktionen wie die Auflistung der \textit{shared libraries} oder der "`indirect symbol table"'. 
\end{description}
$ $\\\\
Für Windows-Phone-Apps:
\begin{description}
	\item[BinScope ] ist Security-Analyse-Tool für Windows-Applikationen. Es wurde von Microsoft für den \textit{Secure Development Lifecicle} entwickelt und steht auf der Microsoft-Webseite\footnote{\url{https://blogs.microsoft.com/microsoftsecure/2012/08/15/microsofts-free-security-tools-binscope-binary-analyzer/}} zur Verfügung.
	\item[BinSkim ] ist der Nachfolger von BinSkim. Jedoch wurden in dieser Masterarbeit mit BinScope oft bessere Ergebnisse erzielt. BinSkim wurde ebenfalls von Microsoft für den \textit{Secure Development Lifecicle} entwickelt und kann über \textit{nuget}\footnote{\url{https://www.nuget.org/packages/Microsoft.CodeAnalysis.BinSkim/}} bezogen werden.
\end{description}
