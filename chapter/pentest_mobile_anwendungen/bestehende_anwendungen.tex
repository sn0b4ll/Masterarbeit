\section{Bestehende Anwendungen}
Trotz der relativ neuen Thematik der mobilen Applikationen gibt es schon einige Programme und Applikationen, die bei der Identifizierung von Schwachstellen helfen können. Im Folgenden sind diese unterteilt in \textit{All-In-One-Frameworks} und Einzelanwendungen. Die Namen sind hierbei sprechend: Sogenannte \textit{All-In-One-Frameworks} bündeln mehrere kleine Anwendungen und automatisieren den Ablauf oder vereinfachen die Bedienung.

\subsection{All-In-One-Framework: MobSF}
\textit{MobSF} ist das einzige, derzeit öffentlich verbreitete \textit{All-In-One-Framework} zur Analyse von mobilen Applikationen auf Open-Source Basis. Diese Plattform dient zur statischen Analyse von \textit{Android-} und \textit{iOS}-Apps sowie zur dynamischen Analyse von \textit{Android}-Apps. Es bündelt viele kleinere Anwendungen, welche unter \ref{Pen:Einzelanwendungen} aufgeführt sind, in einer einfachen Weboberfläche. Es ist Open-Source (GNU General Public License v3.0), in \textit{Python} geschrieben und steht auf \textit{Github} frei zur Verfügung.\footnote{\url{https://github.com/ajinabraham/Mobile-Security-Framework-MobSF}} Die aktuelle Version ist \textit{0.9.4 beta}, wobei teilweise mehrmals pro Woche Code-Änderungen vorgenommen werden.\\

\textit{MobSF} unterstützt die statische Analyse von Apps in den Formaten \textit{android package} (\textit{APK}) und \textit{iPhone Application} (\textit{IPA}) sowie aus einfach komprimierten Archiven (\textit{ZIP}). Zusätzlich beinhaltet \textit{MobSF} einen eingebauten API Fuzzer und ist in der Lage, API-spezifische Schwachstellen wie \textit{XML External Entity} (\textit{XXE}), \textit{Server Side Request Forgery} (\textit{SSRF}) oder \textit{Path Traversal} zu erkennen.

\subsection{Einzelanwendungen}\label{Pen:Einzelanwendungen}
Das \textit{All-In-One-Framework MobSF} greift im Hintergrund oft auf eigenständige Tools zurück. Da es für Penetration-Test oft hilfreich ist, diese ohne ein umgebendes Framework nutzen zu können, sind im Folgenden die wichtigsten Tools kurz aufgeführt.
\\\\
Für Android-Apps:
\begin{description}
	\item[jd-core] ist ein Java Decompiler für Java 5 und spätere Versionen. Er steht zum Download\footnote{\url{http://jd.benow.ca/}} zur Verfügung und kann zum Beispiel über das auf derselben Seite zur Verfügung gestellte \textit{JD-GUI} genutzt werden.
	
	\item[Dex2Jar (d2j)] ist ein Tool zum Umwandeln von \textit{.dex}-Dateien (Dalvik-Bytecode) zu normalem Java-Bytecode (gepackt in einem \textit{Jar}-File). Anschließend können normale Java-Tools zur Analyse verwendet werden. Das Tool ist kostenlos, Open-Source und in Github\footnote{\url{https://github.com/pxb1988/dex2jar}} verfügbar.
	
	\item[enjarify]ist eine modernere Alternative zu \textit{Dex2Jar}.  \textit{enjarify} wurde von Google entwickelt, ist jedoch trotzdem unter der Apache-Lizenz in Github veröffentlicht\footnote{\url{https://github.com/google/enjarify}}.
	
	\item[Dex2Smali] unterstützt ebenfalls bei der Konvertierung von \textit{.dex}-Files in andere Formate. In diesem Fall ist das Zielformat \textit{smali}. Dieses Tool ist ebenfalls Open-Source und in Github\footnote{\url{https://github.com/JesusFreke/smali}} zu finden.
	
	\item[procyon] ist ein Framework zur Analyse von Java-Bytecode. Insbesondere ist ein Decompiler enthalten, welcher den Bytecode in lesbaren Java-Code umwandelt. Das Tool ist kostenlos auf Bitbucket\footnote{\url{https://bitbucket.org/mstrobel/procyon/overview}} verfügbar.
\end{description}
$ $\\
Für iOS-Apps:
\begin{description}
	\item[otool] (auch "`object file displaying tool"' genannt) ist ein Tool zur Analyse von Object-Files. Es ist auf \textit{Mac OS X} bei der Installation von \textit{XCode} enthalten. Es bietet viele brauche Funktionen wie die Auflistung der \textit{shared libraries} oder der "`indirect symbol table"'. 
\end{description}
$ $\\
Für Windows-Phone-Apps:
\begin{description}
	\item[BinScope ] ist ein Security-Analyse-Tool für Windows-Applikationen. Es wurde von Microsoft für den \textit{Secure Development Lifecicle} entwickelt und steht auf der Microsoft-Webseite\footnote{\url{https://blogs.microsoft.com/microsoftsecure/2012/08/15/microsofts-free-security-tools-binscope-binary-analyzer/}} zur Verfügung. Eine genauere Beschreibung ist dem Abschnitt \ref{ref:WeitMobEingTools} zu entnehmen.
	
	\item[BinSkim ] ist der Nachfolger von \textit{BinScope}. Jedoch wurden in dieser Masterarbeit mit \textit{BinScope} oft bessere Ergebnisse erzielt. \textit{BinSkim} wurde ebenfalls von Microsoft für den \textit{Secure Development Lifecicle} entwickelt und kann über \textit{nuget}\footnote{\url{https://www.nuget.org/packages/Microsoft.CodeAnalysis.BinSkim/}} bezogen werden. Eine genauere Beschreibung ist dem Abschnitt \ref{ref:WeitMobEingTools} zu entnehmen.
\end{description}
