\chapter{Einleitung}
Smartphones verbreiten sich immer stärker. Dies zeigt auch eine Statistik von Gartner, nach welcher die Absätze von mobilen Geräten die von herkömmlichen Rechnern und Laptops weiter übertreffen werden \cite{GartnerSales}, siehe Tabelle \ref{ref:GartnerSalesTable}. Mit diesem Trend im Bereich der Verkäufe steigt natürlich auch die Nutzung von Smartphones. Diese können genutzt werden, um Webseiten aus dem Internet aufzurufen, Medien-Inhalte wiederzugeben oder über Kommunikationsprogramme andere Personen zu kontaktieren. Dies sind in Bezug auf Datenschutz und Risiko relativ ungefährliche Anwendungen. Jedoch können auch kritischere Handlungen vollzogen werden, wie zum Beispiel Online-Banking oder das Verwalten von Versicherungs-Verträgen. So erfreuen sich Banking-Apps immer größerer Beliebtheit, da es von Nutzern als praktisch empfunden wird, den Kontostand schnell von unterwegs einsehen zu können oder Bankgeschäfte zu erledigen. Jedoch sollten solche Apps als Konsequenz aus der Kritikalität der Daten so gestaltet sein, dass ein Missbrauch nicht oder nur mit unverhältnismäßig hohem Aufwand möglich ist. Dies ist die Aufgabe der Unternehmen, welche solche Apps bereit stellen. Aktuelle Beispiele\footnote{\url{https://media.ccc.de/v/33c3-7969-shut_up_and_take_my_money}} zeigen jedoch, dass dies viele Unternehmen vor unerwartet hohe Herausforderungen stellt. Mussten bisher nur lokale oder Web-Anwendungen getestet werden, sind es nun Applikationen auf mobilen Geräten mit einem oft wesentlich höheren Anteil an Web-Services und APIs.\\

\begin{figure}[htbp]
	\centering
	\begin{tabular}{ l r r r r}
		Device Type & 2015 & 2016 & 2017 & 2018 \\ \hline
		Traditional PCs (Desk-Based and Notebook) & 244 & 228 & 223 & 216 \\
		Ultramobiles (Premium) & 45 & 57 & 73 & 90 \\
		PC Market & 289 & 284 & 296 & 306 \\
		Ultramobiles (Basic and Utility) & 195 & 188 & 188 & 194 \\
		Computing Devices Market & 484 & 473 & 485 & 500 \\
		Mobile Phones & 1,917 & 1,943 & 1,983 & 2,022 \\
		Total Devices Market & 2,401 & 2,416 & 2,468 & 2,521 \\
	\end{tabular}
	\caption{Worldwide Devices Shipments by Device Type, 2015-2018 (Millions of Units)\cite{GartnerSales}}
	\label{ref:GartnerSalesTable}
\end{figure}

Die Allianz Deutschland AG ist ebenfalls ein Unternehmen, welches sich auf diese Herausforderung vorbereiten muss. Bisher waren viele Wege zum Kunden analog, also per Brief oder direkt über die Außendienstmitarbeiter. Es gab nicht viele elektronische Schnittstellen. Im Jahr 2014 wurde eine Digitalisierungs-Strategie beschlossen und die Web-Anwendung "`Meine-Allianz"' entwickelt. In dieser können Versicherungsnehmer nicht nur Verträge sichten, sondern auch Änderungen an diesen vornehmen. In einer Weiterentwicklung stehen nun auch manche Funktionen in einer iOS-App zur Verfügung. Sollte diese Anwendung eine schwerwiegende Schwachstelle aufweisen, hätte dies für die Allianz nicht nur Schäden in Bezug auf die Reputation, sondern eventuell zusätzliche rechtliche Folgen, da Daten aus Kranken- sowie Lebensversicherungen unter den §203 des StGB fallen. Um dies zu verhindern, sind Prozesse in der Anwendungsentwicklung notwendig, welche die Sicherheit einer Anwendung auf einen möglichst hohen Stand heben.\\

Diese Prozesse umfassen Komponenten wie Source-Code-Scanning, Penetration-Test sowie eine regelmäßig wiederkehrende Prüfung von Anwendungen, selbst nach dem Release.\\

In dieser Arbeit werden als Einführung die Arten von Pen-Tests erläutert. Anschließend werden bestehende, etablierte Prozesse um Komponenten für mobile Applikationen erweitert und Werkzeuge entworfen, um Prozessschritte effizienter abzubilden. Abschließend wird ein Teil eines bereits bestehenden Open-Source-Tools zum automatischen Testen mobiler Anwendungen um Funktionen erweitert, um alle Anforderungen der Allianz Deutschland AG zu erfüllen. Dies umfasst eine Verbesserung der Analysen von Android-, iOS- und Windows-Phone-Apps.