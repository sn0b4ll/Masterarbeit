\newpage
\section{Nachbereitung}
Im Anschluss an die Durchführung des Pen-Tests müssen die Ergebnisse aufbereitet und dem Kunden präsentiert werden. Dies passiert meist in Form eines sogenannten Pen-Test-Reports. Im Folgenden wird auf die Inhalte sowie die Erstellung eingegangen.

	\subsection{Inhalte eines Pen-Test-Berichts}
	Im Folgenden werden die typischen Inhalte eines Pen-Test-Berichts kurz dargestellt. Der Abschnitt ist unterteilt in Allgemeine Informationen (\ref{ref:penBerAllgInf}), Management Summary (\ref{ref:penBerMgmtSum}), Technische Zusammenfassung (\ref{ref:penBerTechZum}) und Findings (\ref{ref:penBerFind}).
	
	\subsubsection{Allgemeine Informationen}\label{ref:penBerAllgInf}
		Der Abschnitt "`Allgemeinen Informationen"' enthält alle organisatorischen Informationen zum Pen-Test. Dies umfasst Autor und Datum des Berichts, den Testzeitraum, die Pen-Tester sowie die  Ansprechpartner im Projekt.
		
	\subsubsection{Management Summary}\label{ref:penBerMgmtSum}
	Das "`Management Summary"' ist, wie der Name schon sagt, als Zusammenfassung für das Management gedacht. Es hat jedoch auch die Funktion, dem Leser kurz die kritischen Erkenntnisse des Pen-Tests zu vermitteln. Dazu ist das Summary in die Abschnitte "`Ausgangssituation"', "`Überblick über die Befunde"' und "`Risikomatrix"' unterteilt.
	
	\begin{description}
		\item[Ausgangssituation] Die Ausgangssituation beschreibt kurz, wie die Anwendung zustande gekommen ist und in welcher Projektphase sich diese befindet. Alternativ oder zusätzlich kann der Grund für den Pen-Test dargestellt werden. 
		
		\item[Überblick über die Befunde] In diesem Abschnitt wird ein kurzer Überblick über die Befunde gegeben, um dem Leser einen erster Eindruck zu vermitteln. Dabei kann die Aufmerksamkeit auf besonders kritische Findings gelenkt werden.
		
		\item[Risikomatrix] Die Risikomatrix stellt die Findings eingeordnet nach Eintrittswahrscheinlichkeit und Schadenspotenzial dar und ermöglicht so dem Leser, sich einen schnellen Überblick über die Verteilung der Findings bezüglich des Gesamtrisikos zu verschaffen.
	\end{description}
		
	\newpage
	\subsubsection{Technische Zusammenfassung}\label{ref:penBerTechZum}
	Die "`Technische Zusammenfassung"' soll wie das "`Management Summary"' einen kurzen Überblicke geben, bezieht jedoch bereits technische Aspekte mit ein und ist ausführlicher als das "`Management Summary"'. Sie ist in vier Abschnitte unterteilt, welche im Folgenden dargestellt werden.
	
	\begin{description}
		\item[Testobjekt] Im Abschnitt	"`Testobjekt"' wird kurz das oder auch die Testobjekte beschrieben. Dabei kann es sich bei Web-Applikation- oder Web-Service-Pen-Tests zum Beispiel um IP-Ranges oder URLs handeln.
		
		\item[Verwendete IPs] Unter "`verwendete IPs"' ist die externe IP der Pen-Tester festgehalten, welche zum Testen genutzt wurde. Diese sollte sich, wie in \ref{ref:vorbInfrastruktur} beschrieben, auf eine Adresse begrenzen.
		
		\item[Zugangsdaten und Benutzerkonten] Ebenfalls werden durch den Auftraggeber festgelegte Zugangsdaten und Benutzerkonten im Bericht dokumentiert. Dabei besteht ein Datensatz im Normalfall aus Benutzername, Passwort und Rollenbeschreibung.
		
		\item[Überblick der Ereignisse] Am Ende der "`Technischen Zusammenfassung"' werden anhand einer Tabelle alle Findings erneut dargestellt. Im Gegensatz zur Risikomatrix sind hier Titel, Kategorie und andere Informationen enthalten.
	\end{description}
		
	\subsubsection{Findings}\label{ref:penBerFind}
	Als letzter Punkt werden die Findings im Detail dargestellt. Dabei werden verschiedene Daten dargestellt, welche im Folgenden kurz aufgeführt werden.
	\begin{description}
		\item[Allgemeines] Anfangs werden die allgemeinen Informationen zu dem Finding dargestellt. Dies umfasst die Nummer, den Namen, die Kategorie sowie den Status des Findings.
		
		\item[Beschreibung] Anschließend an den allgemeinen Teil folgt eine genaue Beschreibung des Findings. Dabei können und sollen durchaus technische Details und Angriffsszenarien dargestellt werden.
		
		\item[Belege] Unter "`Belege"' sollten sämtliche die Beschreibung ergänzenden Screenshots oder Text-Stücke (zum Beispiel die Ausgabe eines Programms) angehängt werden. Sie dienen dazu, das Finding zu erklären und verdeutlichen.
		\item[Kritikalität] In diesem kurzen Teil wird erneut die Eintrittswahrscheinlichkeit, das Schadenspotenzial sowie das daraus resultierende Gesamtrisiko dargestellt.
		
		\item[CVSS] Ergänzend zur Kritikalität wird der \textit{CVSS-Score} (oder falls anders beschlossen, der \textit{DREAD-Score}) mit den einzelnen Metriken und dem \textit{CVSS-Score-Vector} dargestellt.
		
		\item[Empfehlung] Abschließend folgt eine Empfehlung, wie mit dem Finding umgegangen werden sollte.
	\end{description}

	%\subsection{Implementierung in LaTeX}
	%TODO: Implementierung in Latex?	
	
	
	\subsection{Implementierung als Webanwendung}\label{ref:PenProzReportWeb}
	Traditionell werden Pen-Test-Reports in \textit{LaTeX} geschrieben. Dies hat den Vorteil, dass die Reports eine einheitliche Formatierung vorweisen und professionell aussehen. Leider ist die Report-Erstellung oft mühselig und erfordert einigen Aufwand, sodass oft, abhängig von der Länge des Tests, mit mehrere Tage für die Erstellung des Report gerechnet werden muss.\\
	
	Um diesen Aufwand zu minimieren, wurde auch hier eine Web-Anwendung entwickelt. Diese baut auf die gleiche Technologie auf wie die Anwendung, welche für den Fragebogen genutzt wurde (siehe Abschnitt \ref{ref:AufImplInWeb}). So werden Eingaben ebenfalls über eine \textit{HTML5}-Oberfläche aufgenommen und im Hintergrund in ein \textit{LaTeX}-Dokument eingefügt. Anschließend wird das \textit{LaTeX}-Dokument automatisch kompiliert und als \textit{PDF} an den User übergeben.\\
	
	Ein Feature, welches besonders Zeit spart, ist die Möglichkeit, Templates für Findings zu nutzen. So wurden im Rahmen dieser Masterarbeit mehrere Templates für häufig vorkommende Findings angelegt. Einige Beispiele sind:
	\begin{itemize}
		\item Session-Timeout nicht gesetzt
		\item Offene Ports
		\item Fehlende Passwort-Richtlinie
		\item Ungeschützte Cookies
		\item Kein Bruteforce-Schutz im Login
		\item Reflected-XSS in diversen Eingabeformularen
		\item Persistent-XSS in diversen Eingabeformularen
		\item Kein Schutz gegen CSRF
	\end{itemize}
	
	Für jedes Finding-Template wurden jeweils Name, Kategorie, Beschreibung, Empfehlung, Schadenspotenzial, Eintrittswahrscheinlichkeit, Gesamtrisiko sowie einen Standard-\textit{CVSS3-Score} vorbereitend bereits ausgefüllt. Der User kann die Texte und Metriken in der Web-Oberfläche anpassen und Belege hinzufügen.
	