\newpage
\section{Kontinuität}
Ein weiterer wichtiger Teil der Prozesse um Pen-Tests ist Kontinuität. So reicht es nicht, nur einen Pen-Test durchzuführen, wenn eine Anwendung das erste Mal online genommen wird. Denn durch Weiterentwicklung sowie neue entdeckte Angriffsmöglichkeiten kann sich das Sicherheitsniveau ständig ändern.\\

Allgemein gilt, dass das Sicherheitsniveau einer Anwendung immer mindestens der Kritikalität des Anwendung für das Unternehmen entsprechen sollte. Hat eine Anwendung eine hohe Kritikalität, so sollte auch mindestens ein hohes Sicherheitsniveau gefordert werden.\\

Natürlich sollte dafür festgelegt werden, wie die Kritikalität und das Sicherheitsniveau bestimmt werden. Die Kritikalität pro Anwendung ist meist abhängig vom Unternehmen. So würde für einen Automobilhersteller wohl eine Produktionsanlage als hoch kritisch gesehen werden, bei einem Online-Vertrieb wohl eher der Online-Shop. Auch sollte die Angriffsfläche betrachtet werden. So kann man argumentieren, dass eine auf einem abgeschotteten Rechner laufende Anwendung mit hoher Kritikalität eventuell aufgrund der geringen Angriffsfläche mit einem niedrigeren Sicherheitsniveau wie normal betrieben werden darf. Ein weiteres Beispiel wäre eine Anwendung, welche lediglich im Intranet läuft und nicht aus dem Internet erreichbar ist. Für diese Anwendung könnte, aufgrund der geringeren Angriffsfläche, ein geringeres Sicherheitsniveau gefordert werden.\\

Auch das Sicherheitsniveau selbst sowie Maßnahmen zu dessen Aufrechterhaltung müssen im Unternehmen definiert werden. Dabei sollte zuerst die Maßnahmen definiert werden.

\begin{description}
	\item[Pen-Tests] sind eine äußerst effektive, aber kostspielige Maßnahme.
	
	\item[Code-Audits] sind von Personen ausgeführte Analysen des Quellcodes auf Schwachstellen. Da das Personal äußerst gut geschult sein muss, sollten hierfür entweder im Unternehmen Experten eingestellt oder extern gebucht werden. Beim Einkauf von externen Dienstleitern sind die Audits ähnlich teuer wie Pen-Tests, sind aber ebenfalls eine sehr effektive Maßnahme.
	
	\item[Security-Scans] meint automatisierte Sicherheits-Scans durch Software wie \textit{IBM Security AppScan}\footnote{\url{http://www-03.ibm.com/software/products/de/appscan}} oder der \textit{Nessus Vulnerability Scanner}\footnote{\url{https://www.tenable.com/products/nessus-vulnerability-scanner}}. Diese erkennen bekannte Schwachstellen und sind gut automatisierbar. Security-Scanner sind nicht so genau wie Pen-Tests, dafür sind diese, ab einer gewissen Anzahl von Anwendungen, wesentlich günstiger.
	
	\item[Sourcecode-Scans] sind automatisierte Analysen auf Basis des Quellcodes der Anwendung. Produkte wären beispielsweise \textit{Fortify Static Code Analyzer}\footnote{\url{http://www8.hp.com/de/de/software-solutions/static-code-analysis-sast/}} oder \textit{Veracode Static Analysis}\footnote{\url{https://www.veracode.com/products/binary-static-analysis-sast}}. Sourcecode-Scans sind ähnlich effektiv wie Security-Scanner, bedürfen aber, abhängig von der Komplexität der Anwendung, einem größeren Einrichtungsaufwand.
	
\end{description}
 
Hat ein Unternehmen die passenden Maßnahmen aufgebaut, sollten diese über eine zeitliche Einteilung der Maßnahmen zum Sicherheitsniveau vorgenommen werden. Ein frei gewähltes Beispiel dafür ist in Tabelle \ref{tab:PenKontinuitaet} zu sehen. Dabei ist "`Jahr 1"' das erste Jahr, nachdem die Anwendung (nach einem Pen-Test) online genommen wurde.


\begin{table}[h]
	\begin{tabularx}{\textwidth}{l l || p{2.8cm}|p{2.8cm}|p{2.8cm}|p{2.8cm}}
	& \multicolumn{5}{c}{Zeit} \tabularnewline
	 & & Jahr 1 & Jahr 2 & Jahr 3 & Jahr 4 \tabularnewline \cline{2-6}\cline{2-6}
	\multirow{5}{*}{\rotatebox[origin=c]{90}{gef. Sicherheitsniveau}} & Sehr Hoch  & Pen-Test & Pen-Test & Pen-Test & Pen-Test \tabularnewline \cline{2-6}
	& Hoch  & Code-Audit \newline Security-Scan & Pen-Test & Code-Audit & Pen-Test \newline Security-Scan \tabularnewline \cline{2-6}
	& Mittel  & Security-Scan \newline Sourcecode-Scan & Code-Audit & Security-Scan \newline Sourcecode-Scan  & Pen-Test \tabularnewline \cline{2-6}
	& Niedrig  & Security-Scan & Security-Scan & Security-Scan & Security-Scan \tabularnewline \cline{2-6}
	& Sehr Niedrig  & - & Security-Scan & - & Security-Scan \tabularnewline
	\end{tabularx}
		\caption{Einteilung von Maßnahmen zum Sicherheitsniveau über bestimmte Zeiträume} 
	\label{tab:PenKontinuitaet}
\end{table}
